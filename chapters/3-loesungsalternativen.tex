\chapter{Lösungsalternativen}

Die Erstellung von Dokumenten ist wohl schon fast so alt wie die Menschheit
selbst. Es gibt wohl zahllose Methoden, um Dokumente zu erstellen, wobei
das digitale Zeitalter sehr vieles stark vereinfacht hat. Man kann also
in kürzerer Zeit Dokumente mit hoher Qualität erstellen---wenn man es
denn darauf anlegt. Dabei haben sich im digitalen Zeitalter vornehmlich
Satzsysteme bzw. Desktop-Publishing wie WYSIWYG-Programme
ala Microsoft Word, \TeX~oder Adobe InDesign Verbreitung gefunden.

\section{\TeX}

Der Name hat seinen Ursprung aus dem griechischen $\tau\epsilon\chi$,
welches auch die Wurzel für das englische Wort \emph{technology} ist.
$\tau\epsilon\chi$ bedeutet also Technologie, aber auch Kunst.
(\cite{tex-a}, Kapitel 1, Seite 1)

\TeX~ist ein Textsatzsystem, welches für die Erstellung
von qualitativ hochwertigen Büchern ausgelegt ist, mit einem starken Fokus auf
Mathematik.

Es gibt etwa 300 \TeX~Kontrollsequenzen, sog. „Primitive“ welche das
Low-Level TeX bilden. Diese Primitiven sind atomisch und werden nicht weiter
in kleinere Funktionen zerlegt.

Primitiv:

\begin{verbatim}
**\show\input
> \input=\input.
\end{verbatim}

Makro:

\begin{verbatim}
**\show\TeX
> \TeX=macro:
->T\kern -.1667em\lower .5ex\hbox {E}\kern -.125emX.
\end{verbatim}

Zudem kommen noch etwa 600 weitere Kontrollsequenzen „plain \TeX“ dazu,
die zusammen mit den Primitiven das Standard-\TeX~bilden.
(\cite{tex-a}, Kapitel 3, Seite 9--11)

% ALT ===================================
\LaTeX erweitert die Makro-Programmiersprache \TeX um eine Vielzahl von
fertigen Makros. \TeX ist nicht nur eine Makro-Programmiersprache sondern
bietet auch Methoden an, um Typographie zu setzen und Texte auf einer Seite
zu platzieren. Besonders beliebt ist \TeX im mathematischen und
naturwissenschaftlichen Bereich, dank der sehr guten Unterstützung Mathematik
setzen zu können.

Gerade wenn es um Automatisierung geht, ist \LaTeX bisher eine der besten
Lösungen, da \LaTeX in Form von Programmcode geschrieben wird, welcher
z.B. von einem anderen Programm relativ einfach generiert und zusammengestellt
werden kann.

\TeX selbst ist auch eine vollwertige Programmiersprache, wenngleich sie
etwas ungewöhnlich ist, durch die Tatsache, dass sie für den Zweck
Dokumente zu setzen geschaffen wurde.

„Normale“ Programmiersprachen wie C, Java oder Scala wandeln den Code in
maschinenausführbare Instruktionen um; \TeX hingegen wandelt in Quellcode
in ein Schriftsatz-Dokument um. % Quelle: What is Tex?

Die Programmierfähigkeiten sind eher das, was in anderen Sprachen
als Makro bekannt ist. Zudem ist keine echte Standard-Bibliothek wie
bei anderen Programmersprachen, die Dinge wie Datenstrukturen, Zugriff auf
das Betriebssystem etc. bieten vorhanden -- in dieser Hinsicht ist \TeX also
relativ limitiert, zudem ist es der Dokumenten-Generiungstatsache geschuldet,
dass sie Sprache nicht sonderlich elegant ist.

\subsection{Konzepte}

\TeX stellt ein Konzept besonders klar dar, und zwar die \emph{Unterscheidung
zwischen Produktionsformat und Auslieferungsformat.} Diese Unterscheidung
wird von vielen Menschen nicht verstanden oder wahrgenommen; daher kommt
es immer mal wieder vor, dass Microsoft Word Dateien im E-Mail Anhang zu
finden sind.

Bei \TeX ist das Produktionsformat die Quellcode-Dateien und in neuerer
Zeit das Auslieferungsformat eine gesetzte PDF-Datei, welche auf jeder
Plattform gleich angezeigt wird und auch als Archivierungsformat tauglich
ist.

\subsection{Sonnenseiten}

\subsection{Schattenseiten}

\section{Word Processors}


