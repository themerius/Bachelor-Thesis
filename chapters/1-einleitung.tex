\chapter{Einleitung}

Die Idee entstammt der Zeit als ich mein Praxissemester beim Fraunhofer ISE
gemacht habe, da die Tools zur automatischen Dokumentengenerierung, genauer
den Jahresbericht, alle weniger geeignet erschienen. Die gewünschten
Fähigkeiten müssen über mehrere Tools zusammenlaufen, welche nicht immer
ideal zusammenarbeiten; z.B. LaTeX plus Microsoft Word. Ab da an tüftelte
ich daran, wie diesem Umstand Besserung gelingen kann.

Wie schon erwähnt handelt es sich um ein Tool, welches automatische
Dokumentengenerierung ermöglichen soll. Für diese oder ähnliche Aufgaben
gibt es bereits zahlreiche andere Werkzeuge wie z.B.:

\begin{itemize}
  \item LaTeX,
  \item Word, OpenOffice,
  \item Google Docs,
  \item ...
\end{itemize}

Jedes dieser Werkzeuge hat seine individuellen Vor- und Nachteile.
Aber diese Bachelor-Thesis will einen anderen Ansatz ausprobieren, und
seine Machbarkeit, Praxistauglichkeit und Weiterentwicklungsmöglichkeiten
festzustellen bzw. zu überprüfen.
% TODO cite entfernen.
\cite{ref}

\section{Idee}

Wie wäre es, wenn als Dokument-Endprodukt eine HTML/CSS/JS-Webseite
herauskommt? Wenn dieses Endprodukt zudem vom Webbrowser aus passend auf
eine DIN-A4-Seite gedruckt oder als PDF gespeichert werden kann?

Wie wäre es, wenn als Dokumenten-\-Generator-\-Sprache eine \emph{vollwertige}
Programmiersprache zum Einsatz käme? Wenn diese Sprache zudem an die
Domänen\-gege\-ben\-heit die durch den Willen ein Dokument zu verfassen geprägt ist?

Was kann man alles damit anstellen?

\begin{itemize}
  \item Vermischung von statischen und automatisch generierten
        Doku\-menten-\-Be\-stand\-teilen,
  \item Datenaufbereitung quasi zur Laufzeit der Doku\-menten-\-Er\-stell\-ung,
  \item Strukturierungsmöglichkeiten durch den Quellcode, in Pakete, Klassen
        → Objekt-Orientierung,
  \item Webtechnologie ermöglicht dynamische Inhalte,
  \item Webtechnologie ist reaktiv (z.B. auf den Benutzer, Inhalte nachladen),
  \item Gute Kolloberationsmöglichkeiten, Verwaltungsmöglichkeiten,
        da Quellcode
  \item Verknüpfung verschiedener Technologien (Datenbanken, Dateisystem,
        Interpozesskommunikation, etc.),
  \item Sehr flexible Gestaltung des Dokuments, da Webtechnologie möglich,
  \item Webtechnologie ermöglicht Rückkanal, z.B. kollaborierende Benutzer
        können Kommentare schreiben, oder mehr. (Richtung Google Docs.),
  \item Viele Erweiterungsmöglichkeiten, geg. durch Programmiersprache und
        Webtechnologie,
  \item Webtechnologie hat eine sichere Zukunft und ist standardisiert.
\end{itemize}

Es muss also ein kleines JavaScript-Framework entwickelt werden, welches die
Aufgabe der Darstellung des Dokuments übernimmt. Die Zielachitektur.

Zudem braucht es noch ein Programm bzw. Programmiersprache, welches diese
Zielarchitektur füttern kann. Dieses Programm soll Aufgaben wie z.B.
Kapitel-Nummerierung automatisch abwickeln. Weiterhin muss es auch ein
wohlgeformte Schnittstelle zum Benutzer liefern. Diese Kriterien führen
dazu, dass die Entwicklung einer Domänen-Spezifischen Programmiersprache,
kurz DSL, sehr sinnvoll ist.

\section{Eine Wendung}

Als ich soweit an der Thematik getüftelt hatte und die Machbarkeit als
Bachelor-Thesis erkannte trug ich den Vorschlag am ISE vor, jedoch sind
diese kein Informatik Institut und sahen sich nicht in der Lage die Arbeit
zu betreuen, wenngleich sie die Idee sehr nützlich fanden. So wurde mir
nahe gelegt, dass ich bei einem anderen Fraunhofer Institut anklopfen könne.

Ich habe das Fraunhofer SCAI angeschrieben, und Dr. Marc Zimmermann fand
die Idee spannend und auch passend für deren Themengebiet. Sie bereiten u.a.
Patente auf, indem sie eine Patent-PDF-Datei mit Hilfe ihres Java-Framework
zerlegen und die so erhaltenen Daten ggf. mit zusätzlichen Informationen
anreichern. Die Idee von mir hat ihnen sehr zugesagt, da sie noch eine
Möglichkeit suchten, die die aufbereiteten Patente mit Webtechnologie
darzustellen kann bzw. auszuliefern.

\section{Zum Dokument}

In diesem Dokument versuche ich nach Möglichkeit \emph{vollständig}
deutsche Sprache anzuwenden, also nur dort wo es unumgänglich ist
Anglizismen zu verwenden. Das gilt auch gerade für Fachsprache, sofern es
passende deutsche Übersetzungen gibt.
