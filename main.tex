%
% Bachelor Thesis
% Sven Hodapp
%


% -----------
% 1. Präambel
% -----------


% Allgemeine Einstellungen
% ------------------------
\documentclass[
	pdftex,%              PDFTex verwenden da wir ausschliesslich ein PDF erzeugen.
	a4paper,%             Wir verwenden A4 Papier.
	oneside,%             Einseitiger Druck.
	12pt,%                Grosse Schrift, besser geeignet für A4.
	halfparskip,%         Halbe Zeile Abstand zwischen Absätzen.
	%chapterprefix,%       Kapitel mit 'Kapitel' anschreiben.
	headsepline,%         Linie nach Kopfzeile.
	footsepline,%         Linie vor Fusszeile.
	bibtotocnumbered,%    Literaturverzeichnis im Inhaltsverzeichnis nummeriert einfügen.
	idxtotoc%             Index ins Inhaltsverzeichnis einfügen.
]{report}

\usepackage[utf8]{inputenc}
\usepackage[german]{babel}   % deutsche Silbentrennung
\selectlanguage{german}   % damit Table Of Contents Inhaltsverzeichnis genannt wird

\usepackage{geometry}   % Seitenränder einstellbar
\usepackage{textcomp}   % Sonderzeichen, wie Eurosymbol



% Bilder, Farben, farbige Tabellen
% --------------------------------
\usepackage{graphicx, color, colortbl}
\usepackage{longtable}
\usepackage{lscape}
\usepackage{array}       % Erweiterte Tabelleneigenschaften.
%\usepackage{floatflt}   % Bild kann von Text umflossen werden.



% Palatino Schrift
% ----------------
%\usepackage[T1]{fontenc}
%\usepackage[osf]{mathpazo}   % osf aktiviert Mediävalziffern/Minuskelziffern



% Syntax-Highlighting
% -------------------
% Src: http://tihlde.org/~eivindw/latex-listings-for-scala/
\definecolor{dkgreen}{rgb}{0,0.6,0}
\definecolor{gray}{rgb}{0.5,0.5,0.5}
\definecolor{mauve}{rgb}{0.58,0,0.82}

\usepackage{listings}

% "define" Scala
\lstdefinelanguage{Scala}{
  morekeywords={abstract,case,catch,class,def,%
    do,else,extends,false,final,finally,%
    for,if,implicit,import,match,mixin,%
    new,null,object,override,package,%
    private,protected,requires,return,sealed,%
    super,this,throw,trait,true,try,%
    type,val,var,while,with,yield},
  otherkeywords={=>,<-,<\%,<:,>:,\#,@},
  sensitive=true,
  morecomment=[l]{//},
  morecomment=[n]{/*}{*/},
  morestring=[b]",
  morestring=[b]',
  morestring=[b]"""
}

% src: http://lenaherrmann.net/2010/05/20/
%      javascript-syntax-highlighting-in-the-latex-listings-package
\lstdefinelanguage{JavaScript}{
  keywords={
    typeof, new, true, false, catch, function, return, null,
    catch, switch, var, if, in, while, do, else, case, break
  },
  ndkeywords={class, export, boolean, throw, implements, import, this},
  sensitive=false,
  comment=[l]{//},
  morecomment=[s]{/*}{*/},
  morestring=[b]',
  morestring=[b]"
}

\lstdefinelanguage{DSL_ideal}{
  keywords={
    Use, Template, Section, Subsection, Text, PythonScript, Figure
  },
  otherkeywords={named}
}

\lstset{
  frame=tb,
  language=Scala,
  aboveskip=3mm,
  belowskip=3mm,
  showstringspaces=false,
  columns=flexible,
  basicstyle={\fontsize{10}{11}\ttfamily},
  numbers=left,
  numberstyle=\tiny\color{gray},
  keywordstyle=\color{blue},
  commentstyle=\color{dkgreen},
  stringstyle=\color{mauve},
  frame=single,
  breaklines=true,
  breakatwhitespace=true,
  tabsize=2,
  extendedchars=\true,
  inputencoding=utf8,
  escapeinside={\%*}{*)}  % http://tex.stackexchange.com/questions/24528/
}


% Sonstige Pakete
% ---------------
%\usepackage{anysize}   % Seitenränder verändern
%\usepackage{setspace}   % 1.5em Zeilenabstand \begin{onehalfspacing}
\usepackage{bibgerm}   % Anzeigestil des Literaturverzeichnis (gerabbrv)
\usepackage{paralist}  % Individualisierte Aufzählungen


% Definition von globalen Konstanten
% ----------------------------------
\newcommand{\thema}{
  Vergleich von interner und externer DSL-Technologie zur Entwicklung
  eines Textsatzsystems zur automatischen Dokumentengenerierung.}
\newcommand{\schlagworte}{DSL, domain-specific language, Xtext, Scala,
  Dokumentengenerierung, Textsatz, Webtechnologie}
\newcommand{\zusammenfassung}{
  In dieser Arbeit wird ein neuartiges Textsatzsystem entwickelt, welches
als Ziel Webtechnologie generiert und sehr gute Automatisierungsfähigkeiten
besitzen soll. Ein Benutzer soll ein solches Dokument mit einem
Skript, welches auf DSL-Technologie baisert, erstellen können, wobei \TeX~
als Inspirationsquelle dient.

  Dazu wird die externe DSL-Technologie (\emph{Xtext Framework})
mit der internen DSL-Technologie (\emph{Scala Programmiersprache})
verglichen, um herauszufinden welche Technologie am besten zum
Anwendungsfall passt. Die Entscheidung ist auf Scala als interne DSL gefallen.

  Webtechnologie unterstützt von Haus aus \emph{keine} Abstraktion für Seiten,
wie sie insbesondere aus Printdokumenten bekannt ist,
daher ist via JavaScript eine solche Abstraktion implementiert worden.
}
\newcommand{\ausgabedatum}{16.10.2012}
\newcommand{\abgabedatum}{16.01.2013}
\newcommand{\autor}{Sven Hodapp}
\newcommand{\autorStrasse}{Hohentwielstraße 2}
\newcommand{\autorPLZ}{78247 }
\newcommand{\autorOrt}{Hilzingen}
\newcommand{\autorGeburtsort}{Singen/Hohentwiel}
\newcommand{\autorGeburtsdatum}{16.09.1987 }
\newcommand{\prueferA}{Prof. Dr. Marko Boger}
\newcommand{\prueferB}{Dr. Marc Zimmermann}
\newcommand{\firma}{Fraunhofer-Institut für Algorithmen und
Wissenschaftliches Rechnen SCAI}
\newcommand{\studiengang}{Software-Engineering}




% PDF Eigenschaften
% -----------------
\usepackage
[
	colorlinks=false,
	bookmarks = true,
	pdftitle={\thema},
	pdfauthor={\autor},
	pdfsubject={Bachelor Thesis},
	pdfkeywords={\schlagworte},
	urlcolor=blue,
	pdfstartview=FitH
]{hyperref}





% --------------------
% 2. Dokumenten Anfang
% --------------------

\begin{document}

\pagenumbering{roman}

% Deckblatt
% ---------

\begin{titlepage}

\vspace*{-3.5cm}

\begin{flushleft}
\hspace*{-1cm} \includegraphics[width=15.7cm]{titlepages/htwg-logo}
\end{flushleft}

\vspace{2.5cm}

\begin{center}
	\huge{
		\textbf{\thema} \\[4cm]
	}
\normalsize{\textbf{(Working Draft)} \\}
	\Large{
		\textbf{\autor}} \\[4cm]
	\large{
		\textbf{Konstanz, \abgabedatum} \\[1cm]
	}
	\Huge{
		\textbf{{\sf BACHELORARBEIT}}
	}
\end{center}

\end{titlepage}

\include{titlepages/title}
\include{titlepages/abstract}
\include{titlepages/affidavit}



% Inhaltsverzeichnis anzeigen
% ---------------------------
\tableofcontents
\newpage

\pagenumbering{arabic}

% ---------
% 3. Inhalt
% ---------

\chapter{Einleitung}

Die Idee entstammt der Zeit als ich mein Praxissemester beim Fraunhofer ISE
gemacht habe, da die Tools zur automatischen Dokumentengenerierung, genauer
den Jahresbericht, alle weniger geeignet erschienen. Die gewünschten
Fähigkeiten müssen über mehrere Tools zusammenlaufen, welche nicht immer
ideal zusammenarbeiten; z.B. LaTeX plus Microsoft Word. Ab da an tüftelte
ich daran, wie diesem Umstand Besserung gelingen kann.

Wie schon erwähnt handelt es sich um ein Tool, welches automatische
Dokumentengenerierung ermöglichen soll. Für diese oder ähnliche Aufgaben
gibt es bereits zahlreiche andere Werkzeuge wie z.B.:

\begin{itemize}
  \item LaTeX,
  \item Word, OpenOffice,
  \item Google Docs,
  \item ...
\end{itemize}

Jedes dieser Werkzeuge hat seine individuellen Vor- und Nachteile.
Aber diese Bachelor-Thesis will einen anderen Ansatz ausprobieren, und
seine Machbarkeit, Praxistauglichkeit und Weiterentwicklungsmöglichkeiten
festzustellen bzw. zu überprüfen.

\section{Vision}

Wie wäre es, wenn als Dokument-Endprodukt eine HTML/CSS/JS-Webseite
herauskommt? Wenn dieses Endprodukt zudem vom Webbrowser aus passend auf
eine DIN-A4-Seite gedruckt oder als PDF gespeichert werden kann?

Wie wäre es, wenn als Dokumenten-\-Generator-\-Sprache eine \emph{vollwertige}
Programmiersprache zum Einsatz käme? Wenn diese Sprache zudem an die
Domänen\-gege\-ben\-heit die durch den Willen ein Dokument zu verfassen geprägt ist?

\subsection{Gestalterische Fähigkeiten}

Mit Webtechnologien ist es möglich sehr flexible Layouts zu erstellen,
hochwertige Schriftarten einzusetzen und hat eine große Auswahl an
Technologien zu bieten, um z.B. Vektorgrafiken, Animationen oder
andere dynamische und statische Inhalte zu generieren und zu
präsentieren. Es gibt schon sehr viele aktive und geniale Javascript-Bibliotheken,
um Webseiten mit verschiedensten Dingen auszurüsten, wie z.B. MathJAX oder D3.

All diese Webtechnologien sind durch die W3C Organisation standardisiert
und sind quasi schon ubiqutär da nahezu jedes Gerät mittlerweile einen
Webbrowser hat.
Die meisten Softwareentwickler oder auch
Gestalter mit diesen Technologien vertraut, zudem beschäftigen sich
Laien sehr oft auch mit Webtechnologie, allein weil sie z.B. eine kleine
eigene Webseite erstellen wollen.

Bündelung von verschiedenen Webdiensten wird ermöglicht, so könnte man sich
vorstellen bestimmte Informationen über z.B. die Wolfram|Alpha API
anzufordern oder verarbeiten zu lassen oder wenn es nur eine Weiterleitung
als Metainformation ist.
Zusammengefasst:

\begin{itemize}
  \item flexible Layouts,
  \item hochwertige Schriftarten verfügbar,
  \item leichte Erweiterbarkeit,
  \item viele Technologien,
  \item Standardisiert und stark entwickelt → Zukunftssicherheit,
  \item ubiqutär eingesetzt → keine zusätzlichen Softwareinstallationen,
  \item bekannte Materie für viele Menschen,
  \item Verwendung verschiedener Dienste über Web-APIs.
\end{itemize}

\subsection{Interaktive Fähigkeiten}

Webtechnologien haben nicht nur die o.g. Fähigkeiten, sondern gerade
HTML5 glänzt mit seinen interaktiven Fähigkeiten, welche einem relativ
statischen Dokument zusätzliche Möglichkeiten bietet Dinge zu
visualisieren und dennoch druckbar zu halten.

Die neuen visuellen Fähigkeiten sind nicht der einzge Vorteil der sich ergibt,
sondern der Benutzer kann auch viel mehr mit dem Dokument interagieren
und mit anderen Personen die in irgendeiner Form an dem Dokument beteiligt
sind, z.B. durch Kommentare bzw. Diskussionen direkt im Dokument.

Auch können Daten dynamisch nachgeladen werden, z.B. ein Plot der sich
im Dokument bei Änderungen oder neuen Daten automatisch aktualisieren kann.

Anreicherung mit Metadaten ist auch möglich, die Daten können dem Dokument
implizit mitgeliefert werden, so dass Metainformationen über ein Maus-Hover
eingeblendet werden oder eine extra Informationsbox aufgeht oder einfache
Querverweise, z.B. zur komfortablen Referenzierung auf Quellen,
auf andere Dokumente oder Internetadressen.

\subsection{Automatisierungs- und Programmierungsfähigkeiten}

Als Schnittstelle zur Erstellung eines Dokuments soll Programmcode
dienen, ähnlich wie es bei LaTeX und Derivaten der Fall ist, nur
dass eben eine vollwertige Programmiersprache zum Einsatz kommt, diese
dann viele neue Türen öffnet, um die Dokumentengenerierung weiter
zu automatisieren oder mit mehr dynamischen bzw. generierten Inhalten
füllen kann.

\emph{Die Plattform dient zur Kooperation zwischen verschiedensten
Funktionen, Programmen und Diensten.}

Der Programmcode übernimmt auch Dinge wie die automatische Durchnummerierung
von Kapiteln, oder Auflösung von Querreferenzen auf Kapitel, Bilder oder
Literatur. Dadruch, dass es Programmcode ist, kann jedes beliebige Verhalten
hinzugefügt oder modifiziert werden, so dass es auf die Anforderungen
des gewünschten, eventuell speziellen neuartigen, Dokumententypus
zugeschnitten werden kann.

\paragraph{Szenario 0:} Makrdown, Txt, Html (doc?, tex?) \ldots
Bsp. Komplexe und halbautomatische Tabellen.

% TODO schreiben, Word doc integrieren, xls/csv Datei als input als tabelle
% tabelle gleich als datenstruktur nutzbar zur weiterverarbeitung, rechnung,
% anreicherung
% --> alle möglichen technologien kooperieren lassen und auf einen nenner bringen!
% ---> bessere kolloboration (jeder kann sein gewohntes tool nutzen)

\paragraph{Szenario 1:} Immer wenn das Dokument „gebaut“ wird, kann z.B.
automatisch ein Skript angeworfen werden, welches einen Plot als Bild
anfertigt. Dieses Bild kann dann direkt in das resultierende Dokument ohne
Umwege eingebaut werden. Vorteil: Der Dokumenten-Ersteller muss also nicht jedes
mal wenn er den Plot verändert, ihn nochmals manuell anfertigen und
die Bilddatei ins richtige Verzeichnis schieben. → Weniger Handarbeit,
die Dokumenten-Erstellungsumgebung muss nicht verlassen werden.

\paragraph{Szenario 2:} Durch eine vollwertige Programmiersprache hat
der Dokumenten-Ersteller Zugriff auf verschiedene Programm-Bibliotheken,
um z.B. Daten aus einer Datenbank oder dem Dateisystem zu holen, diese
zu verarbeiten, aufzubereiten und im Dokument darzustellen, beispielsweise
als Plot. → Zugriff auf immer aktuelle Daten, mit jeder Dokumenten-Erzeugung.

% TODO Excel Popup (mit VB Script?)  <-> Einbettung im verschiedene Umgebungen.

\paragraph{Szenario 3:} Ermöglicht Konvertierung von verschiedenen
Konventionen auf einen gemeinsamen Nenner, z.B. bei chemischie Formeln
gibt es viele Konventionen der Dateiformale (populär sind SMILES, Molfiles oder
IUPAC-Namen.) Eine Bibliothek könnte all diese Formate annehmen und
für die Webansicht konform machen.

Zudem könnte eine implizite Anreicherung der Informationen vorgenommen werden,
so dass der Benutzer im Programmcode lediglich „Zeichne Strukturformel
für Coffein“ angibt, aber die Bibliothek noch weitere Hintergrundinformationen
zu Coffein aus anderen Quellen (z.B. Wikipedia, Wolfram|Alpha, \ldots) bezieht.

\paragraph{Szenario 4:} Dokumentation
und lauffähiger Code vereint, so dass das Dokument quasi auch schon selbst
die Problemlösung errechnen kann.

Beispiel: Das Dokument beschreibt eine Simulation, zu einem gestellten
Problem. In dem Dokument selbst ist ein Algorithmus beschrieben, der
für die Simulation eingesetzt wird. Aber dadruch, dass das Dokument selbst
in einer vollständigen Programmiersprache geschrieben wird, kann dieser
Algorithmus direkt während der Dokumenten-Generierung ausgeführt werden
und mit Daten aus einer Datenbank gefüttert werden und entsprechende Plots
im resultierenden Dokument darstellen. Also ist der Code, der im Dokument
beschrieben ist auch gleichzeitg der funktionsfähige und ausführbare
Algorithmus.

\paragraph{Szenario 5:} Man könnte sogar so weit gehen, und statt eine einfache
zu HTML-Seite zu generieren einen Webservice starten, der dem Dokument noch
mehr dynamische Fähigkkeiten ermöglicht, z.B. durch bidirektionale
Kommunikation zwischen Webbrowser-Client (Dokumenten-Betrachter) und
Dokumenten-Server, welche eine Interaktion zwischen Benutzer und Dokument
bzw. anderen Benutzern des Dokuments ermöglicht.

\subsection{Zusammengefasst / Was getan werden muss / Was dafür gebraut wird}
% TODO fertig machen
\begin{itemize}
  \item Vermischung von statischen und automatisch generierten
        Doku\-menten-\-Be\-stand\-teilen,
  \item Datenaufbereitung quasi zur Laufzeit der Doku\-menten-\-Er\-stell\-ung,
  \item Strukturierungsmöglichkeiten durch den Quellcode, in Pakete, Klassen
        → Objekt-Orientierung,
  \item Webtechnologie ermöglicht dynamische Inhalte,
  \item Webtechnologie ist reaktiv (z.B. auf den Benutzer, Inhalte nachladen),
  \item Gute Kolloberationsmöglichkeiten, Verwaltungsmöglichkeiten,
        da Quellcode
  \item Verknüpfung verschiedener Technologien (Datenbanken, Dateisystem,
        Interpozesskommunikation, etc.),
  \item Sehr flexible Gestaltung des Dokuments, da Webtechnologie möglich,
  \item Webtechnologie ermöglicht Rückkanal, z.B. kollaborierende Benutzer
        können Kommentare schreiben, oder mehr. (Richtung Google Docs.),
  \item Viele Erweiterungsmöglichkeiten, geg. durch Programmiersprache und
        Webtechnologie,
  \item Webtechnologie hat eine sichere Zukunft und ist standardisiert.
\end{itemize}

Es muss also ein kleines JavaScript-Framework entwickelt werden, welches die
Aufgabe der Darstellung des Dokuments übernimmt. Die Zielachitektur.

Zudem braucht es noch ein Programm bzw. Programmiersprache, welches diese
Zielarchitektur füttern kann. Dieses Programm soll Aufgaben wie z.B.
Kapitel-Nummerierung automatisch abwickeln. Weiterhin muss es auch ein
wohlgeformte Schnittstelle zum Benutzer liefern. Diese Kriterien führen
dazu, dass die Entwicklung einer Domänen-Spezifischen Programmiersprache,
kurz DSL, sehr sinnvoll ist.

\section{Eine Wendung}

Als ich soweit an der Thematik getüftelt hatte und die Machbarkeit als
Bachelor-Thesis erkannte trug ich den Vorschlag am ISE vor, jedoch sind
diese kein Informatik Institut und sahen sich nicht in der Lage die Arbeit
zu betreuen, wenngleich sie die Idee sehr nützlich fanden. So wurde mir
nahe gelegt, dass ich bei einem anderen Fraunhofer Institut anklopfen könne.

Ich habe das Fraunhofer SCAI angeschrieben, und Dr. Marc Zimmermann fand
die Idee spannend und auch passend für deren Themengebiet. Sie bereiten u.a.
Patente auf, indem sie eine Patent-PDF-Datei mit Hilfe ihres Java-Framework
zerlegen und die so erhaltenen Daten ggf. mit zusätzlichen Informationen
anreichern. Die Idee von mir hat ihnen sehr zugesagt, da sie noch eine
Möglichkeit suchten, die die aufbereiteten Patente mit Webtechnologie
darzustellen kann bzw. auszuliefern.

\section{Zum Dokument}

In diesem Dokument versuche ich nach Möglichkeit \emph{vollständig}
deutsche Sprache anzuwenden, also nur dort wo es unumgänglich ist
Anglizismen zu verwenden. Das gilt auch gerade für Fachsprache, sofern es
passende deutsche Übersetzungen gibt.


\chapter{Problembeschreibung → Aufgabenstellung}

Übersicht über das Kapitel.

\section{Zielarchitektur}

\section{Domain-Specific Language}

\section{Verbindung zwischen DSL und Ziel}



%\chapter{Lösungsalternativen}

Die Erstellung von Dokumenten ist wohl schon fast so alt wie die Menschheit
selbst. Es gibt wohl zahllose Methoden, um Dokumente zu erstellen, wobei
das digitale Zeitalter sehr vieles stark vereinfacht hat. Man kann also
in kürzerer Zeit Dokumente mit hoher Qualität erstellen -- wenn man es
denn darauf anlegt. Dabei haben sich im digitalen Zeitalter vornehmlich
Satzsysteme bzw. Desktop-Publishing wie WYSIWYG-Programme
ala Microsoft Word, \LaTeX oder Adobe InDesign Verbreitung gefunden.

\section{\LaTeX}

\LaTeX erweitert die Makro-Programmiersprache \TeX um eine Vielzahl von
fertigen Makros. \TeX ist nicht nur eine Makro-Programmiersprache sondern
bietet auch Methoden an, um Typographie zu setzen und Texte auf einer Seite
zu platzieren. Besonders beliebt ist \TeX im mathematischen und
naturwissenschaftlichen Bereich, dank der sehr guten Unterstützung Mathematik
setzen zu können.

Gerade wenn es um Automatisierung geht, ist \LaTeX bisher eine der besten
Lösungen, da \LaTeX in Form von Programmcode geschrieben wird, welcher
z.B. von einem anderen Programm relativ einfach generiert und zusammengestellt
werden kann.

\TeX selbst ist auch eine vollwertige Programmiersprache, wenngleich sie
etwas ungewöhnlich ist, durch die Tatsache, dass sie für den Zweck
Dokumente zu setzen geschaffen wurde.

„Normale“ Programmiersprachen wie C, Java oder Scala wandeln den Code in
maschinenausführbare Instruktionen um; \TeX hingegen wandelt in Quellcode
in ein Schriftsatz-Dokument um. % Quelle: What is Tex?

Die Programmierfähigkeiten sind eher das, was in anderen Sprachen
als Makro bekannt ist. Zudem ist keine echte Standard-Bibliothek wie
bei anderen Programmersprachen, die Dinge wie Datenstrukturen, Zugriff auf
das Betriebssystem etc. bieten vorhanden -- in dieser Hinsicht ist \TeX also
relativ limitiert, zudem ist es der Dokumenten-Generiungstatsache geschuldet,
dass sie Sprache nicht sonderlich elegant ist.

\subsection{Konzepte}

\TeX stellt ein Konzept besonders klar dar, und zwar die \emph{Unterscheidung
zwischen Produktionsformat und Auslieferungsformat.} Diese Unterscheidung
wird von vielen Menschen nicht verstanden oder wahrgenommen; daher kommt
es immer mal wieder vor, dass Microsoft Word Dateien im E-Mail Anhang zu
finden sind.

Bei \TeX ist das Produktionsformat die Quellcode-Dateien und in neuerer
Zeit das Auslieferungsformat eine gesetzte PDF-Datei, welche auf jeder
Plattform gleich angezeigt wird und auch als Archivierungsformat tauglich
ist.

\subsection{Sonnenseiten}

\subsection{Schattenseiten}

\section{Word Processors}



\chapter{Vergleich und Auswahl}
% ehemals Lösungsweg

In diesen Kapitel wird erörtert, welche Technologie (Xtext oder Scala) für
dieses Projekt am praktikabelsten ist und sich somit schlussendlich durchsetzt.
Kern hierfür ist die Übersicht in Form einer Vergleichsmatrix in Abschnitt
\ref{sec-vergleichsmatrix}.

\section{Warum Scala bzw. Xtext?}\label{sec-warumAusgewaehlt}

\paragraph{Scala}
hat ausgezeichnete Fähigkeiten zur Gestaltung einer
internen DSL, nähere Details siehe Kapitel \ref{sec-grammatikGestaltung},
und als objekt-orientierte sowohl auch funktionale Sprache sehr
vielfältige Möglichkeiten für den Benutzer bietet, egal ob er sich
gerade „innerhalb“ der DSL befindet oder „standard“ Scala schreiben
möchte. Zudem hat Scala eine sehr aktive Community und wird vom
Universitätslehrstuhl unter Prof. Martin
Odersky\footnote{\url{http://lamp.epfl.ch}} weiterentwickelt.
Zudem ist Scala ein OpenSource-Projekt.
Durch die Fähigkeit auf der Java Virtual Machine (JVM) zu laufen,
hat Scala auch eine sehr gute Integration mit Java und
Java-Bibliotheken. (\cite{scala-ref}, Seite 1)

\paragraph{Xtext} ist ein Framework welches auf die Entwicklung externer DSLs
ausgelegt ist und dabei auf die Eclipse IDE Plattform aufbaut
und sehr viel der harten Arbeit abnimmt, was die Entwicklung
externer DSLs nachhaltig vereinfacht.
Auch Xtext läuft auf der JVM und ist stark mit Java integriert, bringt
zudem eine eigene, auch mit Xtext entwickelte, Sprache mit. Diese von der
Syntax her freundlicher als Java ist, aber sich dennoch nahe an den
Java-Konzepten aufhält. Diese Sprache heißt Xtend und wird vornehmlich
intern zur Generator-Programmierung eingesetzt.
Auch Xtext ist ein OpenSource-Projekt und wird hauptsächlich von der
deutschen Firma \emph{itemis AG} betreut bzw. weiterentwickelt. (\cite{xtext})

\section{Vergleichsmatrix}\label{sec-vergleichsmatrix}

In dieser Tabelle ist eine Übersicht über den Vergleich gegeben, indem
die Fähigkeiten bzw. die Möglichkeiten von Xtext als externe DSL und
Scala als interne DSL gegenübergestellt werden.

Für \emph{einige} der gelisteten Fähigkeiten gibt es
tiefergehende Beschreibungen, auf diese
in der ersten Spalte der Tabelle entsprechend referenziert wird.

\begin{landscape}
\begin{longtable}{|p{0.5cm}|p{0.8cm}|p{4.3cm}|p{6.3cm}|p{6.3cm}|}

  \hline
  Nr. & Kap. & Fähigkeit & Xtext (externe DSL) & Scala (interne DSL) \\ \hline \hline
  \endfirsthead

  \hline
  Nr. & Kap. & Fähigkeit & Xtext (externe DSL) & Scala (interne DSL) \\ \hline
  \endhead

  1
  & \ref{sec-grammatikGestaltung}
  & Grammatikalische Gestaltung der DSL
  & {\small Komplett frei und flexibel, da in BNF-Regeln definiert.}
  & {\small Eingeschränkt, man bleibt an Scala's Beschränkungen gebunden, aber
    dennoch sehr ausdrucksstarke Möglichkeiten.}
  \\\hline

  2
  & \ref{sec-gpl}
  & DSL mit General Purpose mischbar
  & {\small Hat viele Hürden, um eine DSL mehr Allgemeingültigkeit zu verpassen.}
  & {\small Alle Scala-Fähigkeiten nativ nutzbar, da die DSL eine normale Library ist.}
  \\\hline

  3
  & \ref{sec-strukturierungsfaehigkeit}
  & Strukturierungsfähigkeit des Codes
  & {\small Muss alles selbst gebaut werden. Vorteil: Es muss nur das nötigste
    umgesetzt werden.}
  & {\small Sämtliche Infrastruktur vorhanden. (Packages, Kontrollstrukturen,
    Build-Tools, ...)}
  \\\hline

  4
  & \ref{sec-erweiterbar}
  & Erweiterbarkeit durch Domain User/Community (z.B. für eigene Templates)
  & {\small Es würde von dem Domain User verlangt werden BNF-Notation zu können,
    Xtend und er wäre auf Eclipse gezwungen.}
  & {\small Einfache Scala Kenntnisse plus eine kleine Anleitung sollten ausreichen,
    die Bindings zu erstellen.}
  \\\hline

  5
  & \ref{sec-erweiterbar}
  & Erweiterbarkeit durch Entwickler
  & {\small Grammatik, Tests und Generator kann nach belieben wachsen, u.a.
    Unterstützung durch Eclipse.}
  & {\small Der Aufwand liegt bei der Entwicklung einer Library. Jedoch müssen
    Testumgebungen etc. selbst eingerichtet werden.}
  \\\hline

  6
  & \ref{sec-erweiterbar}
  & Wiederverwendbarkeit bzw. Kombination mit Vorhandenem
  & {\small Nur eingeschränkt, jedoch sind Grammatik Mixins möglich.}
  & {\small Sehr gut, da Library und mit Scalas Typ- und Vererbungssystem kann nach
    gewohnter Manier kombiniert und erweitert werden.}
  \\\hline

  7
  & \ref{sec-infrastruktur}
  & Sprach-Infrastruktur
  & {\small Xtext generiert automatisch ein speziell angepasstes Eclipse Plugin.}
  & {\small Alles wird mitgeliefert, wie z.B. Compiler, Built-Tools, REPL.
    Breite Unterstützung von vielen Editoren.}
  \\\hline

  8
  & \ref{sec-scalierEinbett}
  & Einbettbarkeit in beliebige Umgebungen
  & {\small De facto Eclipse-Bindung, aber mit individuell angepasstem Eclipse-Plugin,
    welches sich mit dem Projektverlauf automatisch mit anpassen kann.
    Wenn das Ziel ein Arbeitsplatz-Front-End ist, sehr vorteilhaft -- sofern
    Eclipse eingesetzt werden will.}
  & {\small Kann gut in alle möglichen Szenarien eingebettet werden, Benutzung
    innerhalb eines Frameworks möglich, oder Einsatz als Bibliothek,
    Stand-Alone oder in einer Entwicklungsumgebung wie Eclipse denkbar.
    Kann also quasi in eine beliebige Umgebung eingebettet werden
    wo eine JVM läuft oder auch als Service bereitgestellt werden.}
  \\\hline

  9
  & \ref{sec-generator}
  & Generator: Zielplatform
  & {\small Ohne Umwege kann jede Sprache oder Markup aus dem DSL-Modell durch eine
    Template-Engine generiert werden, das Eclipse-Plugin stellt sofort das
    Generat bereit. Jedoch kann nativer Code nicht direkt auf Xtext laufen,
    es muss also ggf. noch ein externer Build o.ä. angestossen werden.}
  & {\small Die DSL selbst kann direkt ein lauffähiges Programm sein. Andere Ziele,
    z.B. andere Programmier-Sprachen oder Markup-Sprachen müssen einen Umweg
    über eine Template-Engine nehmen, das
    Verfahren hierzu muss selbst entwickelt werden (das kann ein Vor- oder
    auch ein Nachteil sein.)}
  \\\hline

  10
  & \ref{sec-generator}
  & Generator: Template-Engine
  & {\small Xtend eine speziell angepasste DSL-Generator-Template-Engine.
    Die BNF-Grammatik wird transparent in Java- bzw. Xtend-Klassen übersetzt,
    mit denen das Ziel über das Template generiert werden kann.}
  & {\small 1. Freie Wahl, z.B. einfache Multiline-Strings, Scala XML oder Scalate;
    wie aus der internen DSL das Ziel generiert wird, benötigt in der Regel
    einen Zwischenschritt (Bindings), welcher programmiert werden muss.
    Scala kann jedoch ggf. das Generat als Unterprogramm ausführen.
    2. Die interne DSL ist selbst lauffähig.}
  \\\hline

  11
  &
  & Entwicklungsaufwand (u.a. Zeit, Einarbeitung)
  & {\small Wenn BNF-Kenntnisse (theoretische Informatik) vorhanden sind,
    relativ leichte Einarbeitung.
    Die Tools nehmen die harte Arbeit ab. Es gibt schon standardisierte
    Vorgehensweisen, z.B. wie der Generator gebaut wird.}
  & {\small Wenn Scala-Kenntnisse vorhanden, ist es mehr oder weniger die Entwicklung
    einer Bibliothek.
    Wie man den Generator baut, muss allerdings überlegt werden.}
  \\\hline

  12
  & \ref{sec-erweiterbar}
  & Software-Lebenszyklus und Wartbarkeit
  & {\small Dank IDE und einer schon eingerichteten Testumgebung, also sehr gut.}
  & {\small Man hat alle Möglichkeiten, die die Scala-Welt bietet, also sehr gut.
    Jedoch ist Handarbeit nötig.}
  \\\hline

  13
  &
  & Tooling (für DSL Gestaltung)
  & {\small Komplette und entsprechend angepasste Eclipse Entwicklungsumgebung.}
  & {\small Die Sprache selbst, sonst keine Hilfen.}
  \\\hline

  14
  & \ref{sec-scalierEinbett}
  & Skalierbarkeit
  & {\small Kommt auf das Generat an. Man ist und bleibt an Eclipse gebunden.}
  & {\small Scala selbst ist in alle Richtungen (Größe, Nebenläufigkeit) sehr gut
    skalierbar.}
  \\\hline

  15
  &
  & DSL als Library bzw. Bereitstellung
  & {\small Ist eine in sich mehr oder weniger geschlossene Struktur.}
  & {\small Interne DSL ist eine ganz normale Scala Library.}
  \\\hline

\end{longtable}
\newpage
\end{landscape}


\subsection{Sprach-Infrastruktur}\label{sec-infrastruktur}

\paragraph{Xtext} wird direkt als fertig eingerichtete Eclipse IDE
ausgeliefert\footnote{\url{http://www.eclipse.org/Xtext/download.html}},
die speziell an die Erstellung von externen DSLs angepasst
ist. Es wird also ein DSL-Erstellungsökosystem „out-of-the-box“ geliefert.

Xtext bietet die folgenden Dinge, mit einer exzellenten IDE Unterstützung:

\begin{itemize}
  \item Grammatik DSL die der erweiterten Backus-Naur Form ähnelt
        (\cite{xtext}, S. 59f),
  \item unterschiedliche Code-Generatoren (siehe. Abschnitt \ref{sec-generator}),
  \item an DSL-Entwicklung angepasste Testumgebung (\cite{xtext}, S. 31.)
\end{itemize}

Zudem sei erwähnt, dass Xtext aus den Grammatik-Regeln automatisch ein
passendes Eclipse-Plugin generiert, welches die Grammatik vollständig
unterstützt, wie z.B. Syntax-Hervorhebung, Überprüfung der grammatikalischen
Korrektheit oder Auto-Vervollständigung.

Weiter werden aus den Grammatik-Regeln Java-Klassen abgeleitet, mit denen
komfortabel die Code-Generierung vorgenommen werden kann. (\cite{xtext}, S. 78)
Der DSL-Programmierer
muss sich also nicht mit abstrakten Syntaxbäumen oder Ähnlichem herumschlagen.

\paragraph{Scala} bringt als vollwertige \emph{General Purpose Language}
ein umfangreiches Ökosystem mit, bestehend aus verschiedenen Werkzeugen
und vielen Bibliotheken, beispielsweise eine mächtige Standard-Bibliothek.
Hervorzuheben sind:

\begin{itemize}
  \item Linker und Compiler,
  \item Built-Tools wie \emph{sbt}\footnote{\url{http://www.scala-sbt.org}}
        mit Abhängigkeits-Management,
  \item Read-Evaluate-Print-Loop (REPL), eine interaktive Scala-Sitzung,
  \item Zugriff auf die Java-Standard-Library,
  \item Umfangreiche Scala-Standard-Library.
\end{itemize}

Darüber hinaus hat Scala sehr gute Fähigkeiten zur Erstellung von internen
DSLs, siehe Abschnitt \ref{sec-grammatikGestaltung}.


\subsection{Strukturierungsfähigkeit}\label{sec-strukturierungsfaehigkeit}

Im Allgemeinen ist mit der Strukturierungsfähigkeit gemeint, wie der
geschriebene (DSL-)Code logisch und sinnvoll gegliedert werden kann,
z.B. durch

\begin{itemize}
  \item Pakete, Module oder Namensräume,
  \item Klassen oder Objekte,
  \item Funktionen bzw. Geltungsbereiche,
  \item eventuell auch Kontrollstrukturen oder Datenstrukturen.
\end{itemize}

Fehlerbehandlung ist auch ein wichtiger Posten, also ob Ausnahmen
geworfen werden können. Hier hat Xtext Eclipse als Helfer, der
z.B. Syntax-Fehler ausfindig machen kann und Scala hat u.a. den Compiler
und Ausnahmen die zur Laufzeit geworfen werden können.

\paragraph{Xtext} ermöglicht es von Grund auf eine Sprache zu entwickeln,
die komplett auf das Domänen-Problem zugeschnitten ist---ohne Kompromisse.
Dadurch dass die Sprache von Grund auf erstellt wird, ist der Entwickler
auch dazu gezwungen sich Gedanken dazu zu machen, wie der DSL-Code
der vom Benutzer geschrieben wird strukturiert werden kann,
was insbesondere bei größeren DSL-Skripten oder
Projekten günstig sein kann---falls gewünscht bzw. gebraucht.
Jedoch ist Handarbeit erforderlich, um solche Fähigkeiten wie sie o.g.
sind in der DSL unter zu bekommen.

\paragraph{Scala} bietet die o.g. Struktuierungsfähigkeiten generisch an,
es muss also keinerlei Aufwand getrieben werden, um die interne DSL mit
diesen Fähigkeiten auszustatten. Soll heißen, all das wird gratis mitgeliefert.


\subsection{General Purpose Language}\label{sec-gpl}

Im Gegensatz zu DSLs stehen \emph{General Purpose Languages}, kurz GPLs.
Das soll heißen, dass mit diesen Sprachen alle Probleme gelöst werden, da
sie turing-vollständig sind.

Hauptunterschiede zwischen DSLs und GPLs sind,

\begin{itemize}
  \item eine DSL zielt auf ein spezifisches Problemfeld ab,
  \item eine DSL enthält Syntax und Semantik, welches sich auf dem gleichen
        Abstraktionslevel wie das der Domäne befindet.
\end{itemize}

(\cite{dsls}, Seite 11)

Das bedeutet, dass das DSL-Skript die unterliegende Implementierung abstrahieren
muss. Es düfen also möglichst keine Implementierungsdetails darin auftauchen.
(\cite{dsls}, Seite 15)

Jetzt kann es sein dass, wie in diesem Projekt gewünscht,
auch die Möglichkeit geboten wird universellen Programmcode zu schreiben,
um z.B. auf das Dateisystem zuzugreifen oder eine Grafik mit einer
externen Bibliothek zu erstellen. Der Benutzer soll also die
Möglichkeit haben, die DSL zeitweise zu verlassen um „normalen“ Programmcode
zu schreiben und danach wieder in die DSL zurückzukehren---ohne große
Aufwand.

Dies ist für dieses Projekt relativ wichtig, dass auch GPL-Elemente zugelassen
werden, um eine möglichst hohe Automatisierung der Dokumentenerstellung
dem Benutzer zu ermöglichen. Siehe dazu als Beispiele
die verschiedenen Szenarien aus Kapitel \ref{sec-idee-szenarien}.

\paragraph{Xtext} bietet die Möglichkeit mittels Xbase GPL-Expressions
(\cite{xtext}, S. 150)
auszuführen---und das obwohl Xtext externe DSLs erstellt, für die es für
gewöhnlich eine sehr harte Arbeit darstellt GPL-Elemente einfließen zu lassen.
Xtext vereinfacht diese harte Arbeit. Aber auch hier ist die Implementierung
nicht ganz kostenfrei, da als Generator nur noch der \emph{JvmModelInferrer}
(siehe Abschnitt \ref{sec-generator}) zum Einsatz kommen kann.
Und auch hier hat man nicht die absolute Freiheit, die man sich eventuell
wünschen würde:

\begin{itemize}
  \item Falls sich der Geltungsbereich von DSL und GPL-Abschnitt überschneiden
        soll, kann es zu Komplikationen kommen, z.B. wenn der DSL-Code
        auf eine Variable aus dem GPL-Abschnitt zugreifen soll (siehe
        Abschnitt \ref{sec-forwardreference}.)
  \item Der JvmModelInferrer kann nur noch Java bzw. die JVM als Ziel haben.
        (\cite{xtext}, S. 148)
\end{itemize}

\paragraph{Scala} ist eine GPL und hier in dem Projekt wird eine interne
DSL mit Scala erstellt. Dort gibt es absolut keine Einschränkungen, es
kann auf den vollen Funktionsumfang von Scala zugegriffen werden.
Da die interne DSL quasi nur eine ausdrucksstarke Scala-Bibliothek ist,
können die Grenzen der DSL mit der GPL verschwimmen.


\subsection{Erweiterbarkeit, Wiederverwendbarkeit und Wartbarkeit}
\label{sec-erweiterbar}

In diesem Projekt ist es sehr wichtig, dass sowohl DSL als auch
die Geschäftslogik flexibel erweiterbar sein soll bzw. sogar
einzelne Teile wiederverwenden zu können.

Warum ist das so wichtig? Weil im späteren Lebenszyklus dieser Software
verschiedenartige Dokumente generierbar sein sollen, wo im Idealfall
die Benutzer selber Änderungen an den Bindungen zwischen
Dokumenten-Template und der DSL vornehmen können. Die Benutzer
wollen eventuell das Dokument um neue Arten von Darstellungen (z.B. spezielle
Tabellen) oder gänzlich neue Möglichkeiten (z.B. ein Chemie-Editor
innerhalb des resultierenden Dokuments) erweitern.
Verschiedene Dokumenten-Arten sollen sich leicht hinzufügen lassen,
z.B. zum einen ein akademischer Bericht, zum anderen ein europäisches Patent.

Eventuell möchte man auch Bestandteile aus z.B. dem Patent-Template in
einem anderen Template wiederverwenden? Das Ziel ist also eine
möglichst lebendige Umgebung, mit der Fähigkeit leicht erweiterbar zu sein.

\paragraph{Xtext} hat hier den Hauptnachteil, dadurch dass quasi nur
Entwickler in der Lage sind die Grammatik zu erweitern oder den
Generator zu pflegen. Aber die Benutzer sollen in der Lage sein
eigene Dokumenten-Templates zu erstellen oder vorhandene zu modifizieren.

\paragraph{Scala} kann hier glänzen dadurch, dass zu jedem Template auch
direkt ein Bindungs-Code vom (versierten) Domänen-Benutzer erstellt
werden kann, da die Komplexitäten in tiefere Schichten gezogen werden können.
Durch das starke Vererbungssystem welches Scala bietet, können solche
Template-Erweiterungen ohne viel unschöne Details erstellt werden, da sich
auf das Wesentliche konzentriert werden kann.


\subsection{Grammatikalische Gestaltung}\label{sec-grammatikGestaltung}

Die grammatikalische Gestaltung ist eine der wichtigsten Eigenschaften
für die Ausdrucksstärke einer DSL und die Ausdrucksstärke ist eines
der wichtigsten Kriterien für die Akzeptanz der Benutzer.

\paragraph{Xtext} ist ein Framework zur Erstellung von externen DSLs und
hat somit naturgemäß ausgezeichnete Fähigkeiten zur Gestaltung einer beliebigen
Grammatik.

Dabei setzt Xtext auf eine selbst entwickelte externe DSL, die sehr große
Ähnlichkeit mit der \emph{Backus-Naur-Form} hat, mit der kontextfreie
Grammatiken bzw. Programmiersprachen entworfen werden können.
(\cite{xtext}, S. 59f)

\paragraph{Xtext Grammatik-Beispiele}

Hier zwei implementierte Beispiel-Grammatiken, um die Flexibilität von
Xtext zu verdeutlichen, einmal eine Grammatik die Ähnlichkeit mit \TeX~
hat und zum anderen eine Grammatik die an die interne DSL von Scala angelehnt,
jedoch entspricht diese nicht ganz dem Endresultat der Scala-DSL, ist.

\begin{lstlisting}[caption=\TeX-ähnliches Xtext-Grammatik-Snippet.]
grammar de.htwg.scaltex.latexdsl.LaTeXDSL
  with org.eclipse.xtext.common.Terminals

generate laTeXDSL "http://www.htwg.de/scaltex/latexdsl/LaTeXDSL"

Model:
  entities += Entity*;

Entity:
  Section | Paragraph;

Section:
  '\\section' '{' content = TEXT '}';

Paragraph:
  content = TEXT;

terminal TEXT  : 
  ( '\\'('b'|'t'|'n'|'f'|'r'|'u'|'"'|"'"|'\\') |
    !('\\'|'{'|'}'|'\n') )*;
\end{lstlisting}

\begin{lstlisting}[label=xtext-gramm,caption=An Scala-DSL angelente Xtext-Grammatik.]
grammar de.htwg.scaltex.dsl.ScalTeX with org.eclipse.xtext.common.Terminals

generate scalTeX "http://www.htwg.de/scaltex/dsl/ScalTeX"

Scaltex:
  'filename' name = ID
  entities += Entity*;

Entity:
  Heading | UniversalEntity | UniversalEntityWithKwargs | ControlCommand;

Heading:
  '§' order = ('>' | '>>' | '>>>') content += STRING;

UniversalEntity:
  '^' name = ID
    content = STRING;

UniversalEntityWithKwargs:
  '^' name = ID
    content = Kwargs;

ControlCommand:
  '!!' content = ID;

Kwargs:
  '('
    (arguments += Argument)*
  ')';

Argument:
  name = ID '=' content = STRING (comma = ',')?;
\end{lstlisting}

\paragraph{Scala} hat für eine Programmiersprache von sich aus schon
sehr gute Fähigkeiten zur Erstellung von internen DSLs, dies
manifestiert sich in folgenden Fähigkeiten (siehe \cite{dsls} Kapitel 6.1):

\begin{itemize}
  \item Unicode-Zeichen in Identifiern erlaubt (\cite{scala-ref}, Seite 3f),
  \item Methoden können als Prefix, Infix oder Postfix Operatoren dienen (\cite{scala-ref}, Kapitel 6.12),
  \item flexible Syntax (z.B. optimale Punkte zum Methodenaufruf),
  \item viel syntaktischer Zucker u.a. mit impliziten Erweiterungen,
  \item Stärken von objektorientierter- und funktionaler-Programmierung
        geschickt vereint,
  \item starke statische Typisierung, jedoch ergänzt der Compiler selbstständig
        sehr viele Typ-Informationen → wirkt fast wie eine dynamisch
        typisierte Sprache.
\end{itemize}

Zudem sei erwähnt, dass es für Scala auch eine Bibliothek gibt mit
der es möglich ist externe DSLs zu entwerfen---diese basiert auf der
\emph{parser combinator} Technik (siehe \cite{dsls} Kapitel 8.)

Im Listing \ref{scala-syntax} ist die wichtigste Essenz aus dem Code
aufgeführt, wie prinzipiell die API in diesem Projekt geformt wird.
Erklärung: Das Objekt O erbt von Areal die Syntax. Bedeutung der Syntax ist
„plus Überschrift Entität, plus Text Entität”. Der Nachteil dabei ist,
dass leider ein + nicht allein für sich stehen kann, da der Compiler dies
als Unäre-Operation zählt und somit nicht wie gewünscht erkennt, ++ jedoch
zählt als Infix-Operation und funktioniert daher wie gewünscht.\footnote{
Diese Erkenntis entstammt aus Antworten auf meine Frage auf Stackoverflow:
\url{http://stackoverflow.com/questions/13367122}.}

\begin{lstlisting}[label=scala-syntax,caption=Scala DSL Syntax (Snippet).]
class EntityBinding {
  // ...
  def § (heading: String) = println(heading)  // new EntityHeading ...
  def txt (text: String) = println(text)  // new EntityText ...
}

class Areal {
  // ...
  val ++ : EntityBinding = new EntityBinding
  // ...
}

// Run Code

object O extends Areal {
  ++ § "My Heading"
  ++ txt "Lorem ipsum ..."
}
\end{lstlisting}

\subsection{Skalierbarkeit und Einbettbarkeit}\label{sec-scalierEinbett}

\paragraph{Xtext} hat eine vollständige und absolute Abhängigkeit
zur Eclipse IDE und kann folglich nicht ohne Eclipse auskommen bzw.
funktionieren. Daher ist man in Sachen Skalierbarkeit bzw. Einbettbarkeit
an die Einschränkungen bzw. auch Vorteile von Eclipse gebunden.

Eclipse ist eine mächtige IDE, die auf mittlere bis große Softwareprojekte
ausgelegt ist, und somit eher unter die schwergewichtigen Entwickler-Werkzeuge
einzuordnen ist. Aber der Vorteil ist, dass selbst riesengroße Projekte
damit verwaltbar sind und das Projekt quasi nach belieben wachsen kann und
durch die Refactoring-Fähigkeiten auch leicht wartbar und erweiterbar ist.
Man könnte also sagen, dass Xtext die Skalierbarkeit von Eclipse erbt.

Das Generat kann je nach eingesetzten Generator (siehe Abschnitt
\ref{sec-generator}) skalieren, da Xtext quasi beliebige Generate erstellen
kann, muss für Einzelfall entschieden werden.

Zur Einbettbarkeit in beliebige Umgebungen, ist der Entwickler bzw.
Domain-Benutzer sehr gebunden und abhängig. Eclipse läuft für gewöhnlich
nur auf Desktop-Computern, also mit grafischer Benutzeroberfläche,
und benötigt relativ vielen Ressourcen, die Eclipse auch
gerne annimmt.
Da Eclipse auch die Generatoren anwirft, um aus der DSL
die Zielarchitektur zu erstellen, ist auch dieser Prozess an Eclipse gebunden.
Daher ist der Einsatz auf z.B. einem dedizierten Server ohne grafische
Benutzeroberfläche nicht so leicht umzusetzen.

Die Einbettbarkeit des Generats ist wieder von seinen eigenen Fähigkeiten
abhängig, also u.U. sehr gut.

\paragraph{Scala} trägt die Skalierbarkeit schon im Namen. 
Als „scalable language“\footnote{
\url{http://www.artima.com/scalazine/articles/scalable-language.html}}
und hat Scala entsprechende Fähigkeiten in alle Richtungen
zu skalieren. Beispiele: Es ist möglich Scala-Programme vom kleinen
Shell-ähnlichen Skript mit 2 Zeilen, bis hin zur Enterprise Software
mit hunderttausenden Zeilen Code zu schreiben. Also vom kleinen mini
Einzelplatzskript bis zu Mainfraimapplikationen mit unzähligen nebenläufigen
Zugriffen.

Scala hat auch keinerlei Bindung zu einer speziellen IDE, aber es gibt
u.a. ein geeignetes Eclipse Plugin\footnote{\url{http://scala-ide.org}}.
Durch diese Bindungslosigkeit, kann
es auch headless oder auf entfernen Rechnern laufen und kann somit auf und für
beliebigen Szenerien entwickelt werden.
Die DSL wird in Form einer normalen Scala-Bibliothek ausgeliefert und
kann quasi in jede Umgebung eingebracht werden, in die auch eine andere
Scala-Bibliothek eingebracht werden kann.

Falls die DSL nicht schon selbst das lauffähige Programm darstellt, gilt
auch hier das selbige wie für das Xtext-Generat: Die Skalierbarkeit und
Einbettbarkeit ist vom Generat bzw. dessen Fähigkeiten abhängig, also u.U.
sehr gut.


\subsection{Generator}\label{sec-generator}

\paragraph{Xtext} hat zwei unterschiedliche Möglichkeiten ein
Generat zu erstellen, zum einen den \emph{Code Generator mit Xtend},
zum anderen den \emph{ModelInferrer mit Xbase}.

Der Code Generator stellt eine Template-Engine bereit, mit der
ein beliebiges Ziel erstellt werden kann, also z.B. C++, XML oder Java.
Dies wird dadurch erreicht, dass via Xtend aus Java-Klassen die aus der
Grammatik von Xtext automatisch generiert wurden ein Template zusammengebaut
werden kann. Man kann also förmlich die DSL entpacken und in ein Template
gießen.
Jedoch muss, je nach Generat, es ggf. kompiliert werden. Der
Domain-Benutzer hat keinen Zugriff auf z.B. Typ-Prüfung der JVM --
das Eclipse-Plugin überprüft lediglich die DSL auf ihre
grammatikalische Korrektheit. Man ist also komplett in der DSL eingesperrt.
Sehr gut also für nicht-JVM-Ziele, die einen strikten Geltungsbereich mit
klar definierten DSL-Vokabeln aufweisen. Kurz: Keine Unterstützung für
Expressions.\cite{xtext}

In Abbildung \ref{fig-igenerator} ist der Code zu sehen, wie eine
Scala-Datei aus der Grammatik aus Listing \ref{xtext-gramm} generiert wird.

\begin{figure}[h!]
  \centering
    \includegraphics[width=0.9\textwidth]{figures/igenerator.png}
  \caption{IGenerator generiert eine scala-Datei aus der Grammatik von Listing \ref{xtext-gramm}.}\label{fig-igenerator}
\end{figure}

Der ModelInferrer zielt direkt auf die JVM bzw. Java ab und ermöglicht es
u.a. JVM-Datentypen direkt mit in die DSL einzubetten, mit allen Möglichkeiten
die Xbase\footnote{Wobei Xbase selbst mit Xtext gebaut wurde. Details auf
\url{http://blog.efftinge.de/2010/09/xbase-new-programming-language.html}}
bietet. Es können also dynamisch Java-Klassen generiert werden, wobei innerhalb
der DSL der Fokus auf das Wesentliche gelegt werden kann und nur dort wo es
wirklich nötig ist können General Purpose Expressions zugelassen werden.
Wobei es anzumerken gilt, dass das nicht ganz transparent geschieht,
es muss ein ModelInferrer geschrieben werden, welcher eben die Java-Klassen
dynamisch generiert.
Dies ist also sehr gut, wenn das Ziel auf Java abgebildet werden soll
und gewollt wird dass auf andere Java-Klassen (wie z.B. java.io.File)
aus der DSL heraus benutzt werden können. Kurz: Unterstützung für Expressions.
Die so entstehenden Java-Dateien müssen aber auch noch vom Domain-Benutzer
kompiliert werden.\cite{xtext}

\paragraph{Scala} hingegen hat die Qual der Wahl was das Templating angeht.
Scala selbst bietet in der Standard-Bibliothek schon
eine Reihe an Möglichkeiten an,
z.B. Multiline-Strings oder XML. Jedoch muss sich der Programmierer erst einmal
eine Architektur überlegen, wie er von den DSL-Klassen auf die Template-Engine
kommt. Weiterhin muss er auch festlegen, wie er das Resultat behandelt,
z.B. wo es gespeichert wird, welche Dateinamen es bekommt etc.
Allerdings hat man für die Automatisierung mehr Werkzeuge in der Hand,
die insbesondere unabhängig von Eclipse sind. So könnte man sich vorstellen
nach dem generieren direkt einen Kompelierungsschritt anzuschließen und
danach das Kompilat auszuführen.

Der große Vorteil einer internen DSL ist jedoch, dass der DSL-Code auch
direkt als Programm laufen kann -- also komplett ohne Umwege, ohne Generierung.
Zudem kann man von der sehr hohen Freiheit und Flexibilität profitieren,
wobei u.U. viel Handarbeit nötig ist.

Auf Abbildung \ref{fig-uml_generator} ist der architekturelle Aufbau
des Generators für den Anwendungsfall zu sehen. Wobei der TemplateStock
lediglich HTML-Snippets für den Builder lagert, der Builder
kennt die im Dokument vorhandenen Areale und wandelt diese in die HTML-Pendante
um, sowie die Entitäten und Seiten die in den Arealen vorkommen werden auch
in ihre JSON-Repräsentation umgewandelt.

\begin{figure}[h!]
  \centering
    \includegraphics[height=0.8\textheight]{figures/uml_generator.pdf}
  \caption{Aufbau des Generators des Anwendungsfalls in Scala.}\label{fig-uml_generator}
\end{figure}

\section{Auswertung und Ergebnis}

Wie schon in Kapitel \ref{par-ablauf} erwähnt,
  fällt hier die Entscheidung welche der beiden
  DSL-Technologien zur Implementierung des Prototypen ausgewählt wird.

In der Tabelle aus Kapitel \ref{sec-vergleichsmatrix} werden Xtext und
Scala gegenübergestellt und bewertet.
Dabei wird bei jeder Fähigkeit bewertet,
  ob Xtext oder Scala die besseren Argumente liefert,
  jedoch beziehen sich die Bewertungen auf das hier vorliegende Projekt
angepasst sind und sich somit nicht 1:1 auf beliebige Projekte anwenden lassen.
Die Bewertungen bzw. Gewichtungen müssen also ggf. für jedes Projekt und dessen
spezifischen Anforderungen individuell angepasst werden, falls gewünscht.

Weiterhin gilt zu beachten, dass die Bewertungen ihrer Natur nach sowohl
\emph{intuitiv} als auch \emph{objektiv} gewichtet sind, da eben vieles vom
Betrachter abhängt und die Fähigkeiten gewisse Überschneidungen haben.
Am Ende steht jedoch ein einigermaßen objektives Maß,
welche Lösung die meisten Vorteile, mit Blick
auf das vorliegende Projekt bietet.

Ich habe mich dazu entschieden eine Skala zwischen \emph{0 und 3} zu verwenden,
wobei 0 schlechte Unterstützung/Möglichkeit/Funktion bzw.
allgemein für das Projekt ungeschickt bedeutet.


\subsection{Auswertungstabelle}\label{sec-auswertungstabelle}

\begin{longtable}{|p{0.5cm}|p{0.8cm}|p{5cm}|c|c|}

  \hline
  Nr. & Kap. & Fähigkeit & Bewertung: Xtext & Bewertung: Scala \\ \hline \hline
  \endfirsthead

  \hline
  Nr. & Kap. & Fähigkeit & Bewertung: Xtext & Bewertung: Scala \\ \hline
  \endhead

  1
  & \ref{sec-grammatikGestaltung}
  & Grammatikalische Gestaltung der DSL
  & 3
  & 1
  \\\hline

  2
  & \ref{sec-gpl}
  & DSL mit General Purpose mischbar
  & 1
  & 3
  \\\hline

  3
  & \ref{sec-strukturierungsfaehigkeit}
  & Strukturierungsfähigkeit des Codes
  & 1
  & 3
  \\\hline

  4
  & \ref{sec-erweiterbar}
  & Erweiterbarkeit durch Domain User/Community (z.B. für eigene Templates)
  & 0
  & 3
  \\\hline

  5
  & \ref{sec-erweiterbar}
  & Erweiterbarkeit durch Entwickler
  & 3
  & 2
  \\\hline

  6
  & \ref{sec-erweiterbar}
  & Wiederverwendbarkeit bzw. Kombination mit Vorhandenem
  & 1
  & 2
  \\\hline

  7
  & \ref{sec-infrastruktur}
  & Sprach-Infrastruktur
  & 3
  & 3
  \\\hline

  8
  & \ref{sec-scalierEinbett}
  & Einbettbarkeit in beliebige Umgebungen
  & 2
  & 3
  \\\hline

  9
  & \ref{sec-generator}
  & Generator: Zielplatform
  & 3
  & 3
  \\\hline

  10
  & \ref{sec-generator}
  & Generator: Template-Engine
  & 3
  & 2
  \\\hline

  11
  &
  & Entwicklungsaufwand (u.a. Zeit, Einarbeitung)
  & 2
  & 2
  \\\hline

  12
  & \ref{sec-erweiterbar}
  & Software-Lebenszyklus und Wartbarkeit
  & 2
  & 2
  \\\hline

  13
  &
  & Tooling (für DSL Gestaltung)
  & 3
  & 0
  \\\hline

  14
  & \ref{sec-scalierEinbett}
  & Skalierbarkeit
  & 1
  & 3
  \\\hline

  15
  &
  & DSL als Library bzw. Bereitstellung
  & 1
  & 3
  \\\hline

\end{longtable}


\subsection{Ergebnis}

Wenn man die Werte aus der Tabelle \ref{sec-auswertungstabelle} zusammenrechnet
kommt ein Ergebnis zustande, indem sich \textbf{Scala mit 35 zu
29 Punkten in Vergleich mit Xtext} durchsetzt.

Somit ist für \emph{dieses} Projekt Scala besser geeignet.
Und wird als eigentliche Implementierungsbasis für den Prototyp verwendet.


\chapter{Architektur}

Architektureller Aufbau der Softwaresysteme.

\section{Vorwärtsverweis „Forward Reference“}

Dieses Problem tritt auf, wenn z.B. in einem Text auf eine Abbildung
verweisen wird, welche innerhalb des Programmflusses erst später verfügbar
wird.

\begin{lstlisting}
... auf Abbildung {Reference} ist zu sehen ...

Reference = new Figure(...)
\end{lstlisting}

Hier wird also auf \lstinline|Reference| bereits zugegriffen,
bevor sie überhaupt existiert. Erschwerend kommt hinzu, dass sich die
Reihenfolge der Entitäten (Texte, Bilder, etc.) nicht verändert werden darf.
Die Abbilung soll also an der Position im Dokument erscheinen, an der sie
auch im Dokument-Quellcode geschrieben wurde, da es sich hier um eine DSL
handelt, die sich möglichst nahme am eigentlichen Dokument orientiert.

% Note:
% Von oben nach unten, wir die Lösung immer unflexibler, unsicherer und
% aufwendiger in der Implementierung. Wobei Request-Queue und Pattern-Matching
% etwas zusammengehören.


\subsection{Closure}

Insbesondere funktionale Programmiersprachen wie Scala haben die
Möglichkeit Closures zu bilden, d.h. der Teile Geltungsbereich (Scope)
der äußeren Funktion  kann von der inneren Funktion beibehalten werden,
auch wenn der Geltungsbereich der äußeren Funktion bereits verwirkt ist.
% TODO(oder in appendix?)

\begin{lstlisting}
def outer_func = {
  val v = 15
  (x: Int) => x + v  // lambda function
}
\end{lstlisting}

Auf \lstinline|v| kann noch über die Lambda-Funktion zugegriffen werden,
selbst wenn \lstinline|outer_func| nicht mehr exisiert. Teile des
\lstinline|outer_func|-Geltungsbereichts werden quasi mitgezogen.

Genau mit dieser Technik kann man den Vorwärtsverweis in den Griff bekommen.
Es wird der äußere Geltungsbereich eines Objects nach innen gezogen,
um dort später wenn alle Entitäten bekannt sind und die Referenz aufgelöst
wurde darauf zuzugreifen.

\begin{lstlisting}
object O {
  val text = () => s"… auf Abbildung $reference ist zu sehen …"
  val reference = 3
}

O.text()  // wenn reference vorhanden
\end{lstlisting}

Hier wird zudem die Eigenschaft des \lstinline|object| ausgenutzt,
dass die im \lstinline|object| genannten Variablen immer schon vom Compiler
zumindest mit einer \lstinline|null|-Referenz exisieren, aber die Reihenfolge
der eigentlichen Instanziierung wird nicht verändert. Durch das
\lstinline|lazy|-Keyword von Scala, würde die Reihenfolge modifiziert werden
und ist dadurch nicht verwendbar.

Nachteil hier ist, dass der Domänen-Benutzer innerhalb der DSL diese
\lstinline|() => s""|-Magie schreiben müsste -- was zu Verwirrung und
Unverständnis führen würde.


\subsubsection{Verbesserung für den Domänen-Benutzer}

Ideal wäre also nun eine Lösung inder der Domänen-Benutzer keine
Aufmerksamkeit auf die Closure-Magie verschwenden muss.

Wie bereits in einem vorherigen Kapitel % TODO(oder in appendix?)
erwähnt wird auf den ab Scala 2.10 verfügbaren
\lstinline|StringContext| (\lstinline|s"…"|) gesetzt.
Dieser lässt sich so erweitern, dass
sich die Closure-Magie verstecken lässt und somit in die API
gezogen wird und der Domänen-Benutzer davon gar nichts mitbekommt.

\begin{lstlisting}
implicit def byname_to_noarg[A](a: => A) = () => a

case class StringContext(parts: String*) {
  def $ (args: (() => Any)*) = () => {
    val unpacked_args = args.map(a => a())
    scala.StringContext(parts: _*).s(unpacked_args: _*)
  }
}

object O {
  val text = $"… auf Abbildung $reference ist zu sehen …"
  val reference = 3
}

O.text()  // wenn reference vorhanden
\end{lstlisting}

\lstinline|byname_to_noarg| ist eine implizite Konvertierung von einem
beliebigen Typ \lstinline|A| mit Call-by-Name
% TODO(fußzeile, appendix "benutzte scala technologien"?)
\lstinline|a: => A| in eine Lambda-Funktion \lstinline|() => a|.

\lstinline|def $| fügt die Möglichkeit hinzu \lstinline|$"…"| als individuell angepassten \lstinline|StringContext| zu verwenden. Es wird eine variable 
Argumentenliste mit den implizit zu Call-by-Name konvertierten Argumenten
aus einem beliebigen \lstinline|$"…"|-String übergeben und in ein Closure gepackt,
welches erst dann ausgeführt wird, wenn die Referenzen auch tatsächlich
vorhanden sind.

% http://stackoverflow.com/questions/13307418/scala-variable-argument-list-with-call-by-name-possible

% http://stackoverflow.com/questions/13270906/cast-scala-string-to-stringcontext-and-virtually-forward-references

\subsection{Eval}

\subsection{Datenstrukturbasierend}

\subsubsection{Request-Queue}

Nur Idee!

\subsubsection{Pattern-Matching}

Nur Idee!

\subsubsection{„place here“}

Praktikabel mit Xtext-Generator. Prototypisierung ausnutzen (C-Header).

\chapter{Zusammenfassung}


\section{Problemstellung}

Erstellung qualitativ hochwertiger Dokumente war,
ist und bleibt eine Herausforderung.
Das hier erforschte Textsatzsystem versucht die ubiquitäre und hoch entwickelte
Webtechnologie für die Darstellung klassischer Dokumente bzw. Printdokumente
zu verwenden.
Mit Vorteilen ausgerüstet wie hoher gestalterischer Freiheit,
interaktiven Inhalten und aktuellster Technik ist Webtechnologie
mittlerweile\footnote{
HTML5-Technologien haben eine immense Qualitätssteigerung einher gebracht.}
eine sehr gute Basis zur Darstellung von hochwertigen Dokumenten.


\section{Ziel}

Mithilfe einer DSL soll der Dokumentenersteller eine intuitive Methode
bekommen, um ein Dokument auf Basis von Webtechnologie zu setzen.
Der Klassiker \TeX~kann auch als eine DSL zur Dokumentengenerierung
angesehen werden und dient als Inspiration.

Bevor jedoch das Textsatzsystem implementiert werden kann, muss eine
geeignete DSL-Technologie gefunden werden.
Dazu wurden die zwei DSL-Technologien, Details in Kapitel \ref{sec-vergleich},
verglichen:
Das Xtext-Framework zur Erstellung \emph{externer DSLs} und
die Scala-Programmiersprache zur Erstellung \emph{interner DSLs}.


\section{Vorgehen}

Zum Vergleich der zwei DSL-Technologien sind
in der Vergleichsmatrix in Kapitel \ref{sec-vergleichsmatrix}
15 Fähigkeiten aufgeführt, jede Fähigkeit enhält je eine Beschreibung
wo Xtext mit Scala gegenübergestellt wird.

Die Zielarchitektur ist Webtechnologie, für diese ein JavaScript-Framework
entwickelt wurde, welches eine Abstraktion für Seiten einführt, so dass eine
Webseite Eigenschaften und Aussehen eines Printdokument annehmen kann.

Es konnte gezeigt werden,
dass eine intuitive DSL zur Dokumentengenerierung, inspiriert von \TeX,
auch mit Scala vernünftig umsetzbar ist---ein ausführliches Beispiel
für ein DSL-Skript ist im Anhang \ref{sec-api-resultat} und
nähere Erläuterungen dazu in Kapitel \ref{sec-api-design}.


\section{Ergebnisse}

\paragraph{Vergleich.}
Mit Scala als Gewinner des Vergleichs wird der DSL-Teil des Textsatzsystems als
interne DSL implementiert. Da eine interne Scala-DSL einer Softwarebibliothek
entspricht, sind somit alle Scala-Programmierfähigkeiten im DSL-Skript nutzbar.

Scala hat eine starke Sprachinfrastruktur insbesondere liefert Scala eine
ausgereifte Code-Strukturierung mit.

Scala kann es einem DSL-Benutzer ermöglichen, selbst Dokumententemplates
zu schreiben.

Elegante Lösung des Vorwärtsverweisproblems möglich---siehe Kapitel
\ref{sec-forwardreference}.

Dieser Vergleich kann auch allgemeinen für \emph{externe versus
interne DSLs} gelten, wenn die Xtext- bzw. Scala-spezifischen Besonderheiten
außen vor gelassen werden.
Da das Ergebnis auf den Anwendungsfall „Textsatzsystem“ getrimmt ist muss
ggf. eine Anpassung, insbesondere bei der Bewertung, vorgenommen werden,
falls die Ergebnisse für andere Projekte adaptiert werden sollen.

\paragraph{Zielarchitektur.}
Um eine Webseite wie ein Printdokument aussehen zu lassen wurde ein
JavaScript-Framework erstellt, welches folgende Ziele erreicht:

\erreichtZielarchi

Während der Entwicklung entstanden u.a. diese Dokumententemplates: (1)
der Fraunhofer Bericht und (2) die Patentschrift des europäischen
Patentregisters.

\paragraph{DSL und Generator.} 
Es wurde eine Scala-DSL und ein eingebauter Generator entwickelt,
die folgenden Ziele umsetzt werden:

\erreichtDSL

\emph{Anmerkung:}
Dieses System wird vom Fraunhofer SCAI für die Aufbereitung von
chemischen Patentschriften eingesetzt.


\section{Ausblick}

Dieses Projekt kann in viele Richtungen weiterentwickelt werden werden,
wie in Abschnitt \ref{sec-idee} näher erläutert---hier einige Beispiele:

\begin{itemize}
  \item Die DSL kann erweitert werden,
        um eine Syntax für z.B. Tabellen zu schaffen.
  \item Weitere Logik Klassen die z.B. ermöglichen verschiedene Formate wie
        Markdown zu importieren und intern in Entitäten umzuwandeln.
  \item Kommunikation mit einem Webserver (via JavaScript) für mehr Interaktion,
        z.B. für Benutzerkommentare im Dokument oder Daten die sich automatisch
        aktualisieren.
  \item Neuartige Templates, die z.B. ein Dokument als Mindmap
        darstellen lassen oder aus dem Dokument eine Präsentation erstellen.
  \item Die Templates brauchen noch striktere Paradigmen, falls mehr
        Templates gewünscht werden, so dass sie leicht austauschbar bleiben.
        Beispielsweise jedes Template hat ein vier Spaltenlayout und n definierte
        Zeilen.
\end{itemize}

\begin{appendix}
  \chapter{Einrichtung und Betrieb}

GitHub Repositoria:

\begin{itemize}
  \item Zielarchitetkur: \url{https://github.com/themerius/ScalTeX-templates}
  \item DSL: \url{https://github.com/themerius/ScalTeX-DSL}
\end{itemize}

\section{Voraussetzungen}

\begin{itemize}
  \item Java Virtual Machine bzw. OpenJDK in Version 6 oder 7,
  \item UNIX Betriebssystem,
  \item git,
  \item sbt.
\end{itemize}

\section{Installation}

\begin{enumerate}
  \item Mit git das Repositorium herunterladen:
        \begin{verbatim}
          git clone https://github.com/themerius/ScalTeX-DSL
        \end{verbatim}
  \item In das \verb+ScalTeX-DSL/Scala-NextGen+ Verzeichnis wechseln,
  \item \verb+sbt+ ausführen,
  \item in der sbt-shell den Befehl \verb+run+ eingeben.
  \item (Mit \verb+test+ werden die Testszenarien ausgeführt.)
\end{enumerate}

Nach dem \verb+run+ Befehl werden von sbt automatisch die benötigten
Bibliothektsabhänigkeiten und korrekte Scala Version heruntergeladen.
Danach werden die Quellen kompiliert und ausgeführt---es sollte dann eine
\verb+_output/output.html+ Datei entstehen, mit dem Dokument als HTML-Datei,
welche das kleine hier im Dokument entstandene JavaScript-Framework verwendet.

In \verb+src/main/scala/main.scala+ liegt das
beispielhafte Dokumentenskript.
\end{appendix}

\addcontentsline{toc}{chapter}{Literaturverzeichnis}
\bibliographystyle{apalike}  % apalike
\bibliography{bib/references}

%\listoffigures
%\listoftables

\end{document}

