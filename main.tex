%
% Bachelor Thesis
% Sven Hodapp
%


% -----------
% 1. Präambel
% -----------


% Allgemeine Einstellungen
% ------------------------
\documentclass[
	pdftex,%              PDFTex verwenden da wir ausschliesslich ein PDF erzeugen.
	a4paper,%             Wir verwenden A4 Papier.
	oneside,%             Einseitiger Druck.
	12pt,%                Grosse Schrift, besser geeignet für A4.
	halfparskip,%         Halbe Zeile Abstand zwischen Absätzen.
	%chapterprefix,%       Kapitel mit 'Kapitel' anschreiben.
	headsepline,%         Linie nach Kopfzeile.
	footsepline,%         Linie vor Fusszeile.
	bibtotocnumbered,%    Literaturverzeichnis im Inhaltsverzeichnis nummeriert einfügen.
	idxtotoc%             Index ins Inhaltsverzeichnis einfügen.
]{report}

\usepackage[utf8]{inputenc}
\usepackage[german]{babel}   % deutsche Silbentrennung
\selectlanguage{german}   % damit Table Of Contents Inhaltsverzeichnis genannt wird

\usepackage{geometry}   % Seitenränder einstellbar
\usepackage{textcomp}   % Sonderzeichen, wie Eurosymbol



% Bilder, Farben, farbige Tabellen
% --------------------------------
\usepackage{graphicx, color, colortbl}
\usepackage{longtable}
\usepackage{lscape}
\usepackage{array}       % Erweiterte Tabelleneigenschaften.
%\usepackage{floatflt}   % Bild kann von Text umflossen werden.



% Palatino Schrift
% ----------------
%\usepackage[T1]{fontenc}
%\usepackage[osf]{mathpazo}   % osf aktiviert Mediävalziffern/Minuskelziffern



% Syntax-Highlighting
% -------------------
% Src: http://tihlde.org/~eivindw/latex-listings-for-scala/
\definecolor{dkgreen}{rgb}{0,0.6,0}
\definecolor{gray}{rgb}{0.5,0.5,0.5}
\definecolor{mauve}{rgb}{0.58,0,0.82}

\usepackage{listings}

% "define" Scala
\lstdefinelanguage{Scala}{
  morekeywords={abstract,case,catch,class,def,%
    do,else,extends,false,final,finally,%
    for,if,implicit,import,match,mixin,%
    new,null,object,override,package,%
    private,protected,requires,return,sealed,%
    super,this,throw,trait,true,try,%
    type,val,var,while,with,yield},
  otherkeywords={=>,<-,<\%,<:,>:,\#,@},
  sensitive=true,
  morecomment=[l]{//},
  morecomment=[n]{/*}{*/},
  morestring=[b]",
  morestring=[b]',
  morestring=[b]"""
}

\lstset{
  frame=tb,
  language=Scala,
  aboveskip=3mm,
  belowskip=3mm,
  showstringspaces=false,
  columns=flexible,
  basicstyle={\small\ttfamily},
  numbers=left,
  numberstyle=\tiny\color{gray},
  keywordstyle=\color{blue},
  commentstyle=\color{dkgreen},
  stringstyle=\color{mauve},
  frame=single,
  breaklines=true,
  breakatwhitespace=true,
  tabsize=2,
  extendedchars=\true,
  inputencoding=utf8
}


% Sonstige Pakete
% ---------------
%\usepackage{anysize}   % Seitenränder verändern
%\usepackage{setspace}   % 1.5em Zeilenabstand \begin{onehalfspacing}
\usepackage{bibgerm}   % Anzeigestil des Literaturverzeichnis (gerabbrv)
\usepackage{paralist}  % Individualisierte Aufzählungen


% Definition von globalen Konstanten
% ----------------------------------
\newcommand{\thema}{
	Vergleich von interner und externer DSL-Technologie zur Entwicklung
	eines Textsatzsystems zur automatischen Dokumentengenerierung.}
\newcommand{\schlagworte}{DSL, Dokumentengenerierung}
\newcommand{\zusammenfassung}{
	Im Zuge dieser Bachelor-Thesis wird ein zeitgemäßer Ansatz für ein Texsatzsystem entwickelt und auf die Probe gestellt, als Vorbild dient TeX.

	Kerntechnologien sind HTML5, CSS3 und Javascript zur Gestaltung und Darstellung eines Dokuments, das dabei entstehende Framework ermöglicht semi-automatische Dokumentengenerierung. Zur Generierung der eigentlichen Dokumenteninhalte werden zwei verschiedene DSL Ansätze Verglichen, zum einen eine interne Scala-DSL, zum anderen eine externe Java-DSL (Xtext).

	Vorteile gegenüber TeX sollen sein: Flexibles Dokument-Templating und interaktive Dokument-Elemente, ermöglicht durch die ubiquitäre HTML-Technologie. Ein moderneres und mächtigeres Sprachkonzept als es LaTeX bietet, mit allen Finessen einer kompletten und verbreiteten Programmiersprache.}
\newcommand{\ausgabedatum}{16.10.2012}
\newcommand{\abgabedatum}{16.01.2013}
\newcommand{\autor}{Sven Hodapp}
\newcommand{\autorStrasse}{Hohentwielstraße 2}
\newcommand{\autorPLZ}{78247 }
\newcommand{\autorOrt}{Hilzingen}
\newcommand{\autorGeburtsort}{Singen/Hohentwiel}
\newcommand{\autorGeburtsdatum}{16.09.1987 }
\newcommand{\prueferA}{Prof. Dr. Marko Boger}
\newcommand{\prueferB}{Dr. Marc Zimmermann}
\newcommand{\firma}{Fraunhofer SCAI}
\newcommand{\studiengang}{Software-Engineering}




% PDF Eigenschaften
% -----------------
\usepackage
[
	colorlinks=false,
	bookmarks = true,
	pdftitle={\thema},
	pdfauthor={\autor},
	pdfsubject={Bachelor Thesis},
	pdfkeywords={\schlagworte},
	urlcolor=blue,
	pdfstartview=FitH
]{hyperref}





% --------------------
% 2. Dokumenten Anfang
% --------------------

\begin{document}

\pagenumbering{roman}

% Deckblatt
% ---------

\begin{titlepage}

\vspace*{-3.5cm}

\begin{flushleft}
\hspace*{-1cm} \includegraphics[width=15.7cm]{titlepages/htwg-logo}
\end{flushleft}

\vspace{2.5cm}

\begin{center}
	\huge{
		\textbf{\thema} \\[4cm]
	}
\normalsize{\textbf{(Working Draft)} \\}
	\Large{
		\textbf{\autor}} \\[4cm]
	\large{
		\textbf{Konstanz, \abgabedatum} \\[1cm]
	}
	\Huge{
		\textbf{{\sf BACHELORARBEIT}}
	}
\end{center}

\end{titlepage}

\include{titlepages/title}
\include{titlepages/abstract}
\include{titlepages/affidavit}



% Inhaltsverzeichnis anzeigen
% ---------------------------
\tableofcontents

\pagenumbering{arabic}

% ---------
% 3. Inhalt
% ---------

\chapter{Einleitung}

Die Idee entstammt der Zeit als ich mein Praxissemester beim Fraunhofer ISE
gemacht habe, da die Tools zur automatischen Dokumentengenerierung, genauer
den Jahresbericht, alle weniger geeignet erschienen. Die gewünschten
Fähigkeiten müssen über mehrere Tools zusammenlaufen, welche nicht immer
ideal zusammenarbeiten; z.B. LaTeX plus Microsoft Word. Ab da an tüftelte
ich daran, wie diesem Umstand Besserung gelingen kann.

Wie schon erwähnt handelt es sich um ein Tool, welches automatische
Dokumentengenerierung ermöglichen soll. Für diese oder ähnliche Aufgaben
gibt es bereits zahlreiche andere Werkzeuge wie z.B.:

\begin{itemize}
  \item LaTeX,
  \item Word, OpenOffice,
  \item Google Docs,
  \item ...
\end{itemize}

Jedes dieser Werkzeuge hat seine individuellen Vor- und Nachteile.
Aber diese Bachelor-Thesis will einen anderen Ansatz ausprobieren, und
seine Machbarkeit, Praxistauglichkeit und Weiterentwicklungsmöglichkeiten
festzustellen bzw. zu überprüfen.

\section{Vision}

Wie wäre es, wenn als Dokument-Endprodukt eine HTML/CSS/JS-Webseite
herauskommt? Wenn dieses Endprodukt zudem vom Webbrowser aus passend auf
eine DIN-A4-Seite gedruckt oder als PDF gespeichert werden kann?

Wie wäre es, wenn als Dokumenten-\-Generator-\-Sprache eine \emph{vollwertige}
Programmiersprache zum Einsatz käme? Wenn diese Sprache zudem an die
Domänen\-gege\-ben\-heit die durch den Willen ein Dokument zu verfassen geprägt ist?

\subsection{Gestalterische Fähigkeiten}

Mit Webtechnologien ist es möglich sehr flexible Layouts zu erstellen,
hochwertige Schriftarten einzusetzen und hat eine große Auswahl an
Technologien zu bieten, um z.B. Vektorgrafiken, Animationen oder
andere dynamische und statische Inhalte zu generieren und zu
präsentieren. Es gibt schon sehr viele aktive und geniale Javascript-Bibliotheken,
um Webseiten mit verschiedensten Dingen auszurüsten, wie z.B. MathJAX oder D3.

All diese Webtechnologien sind durch die W3C Organisation standardisiert
und sind quasi schon ubiqutär da nahezu jedes Gerät mittlerweile einen
Webbrowser hat.
Die meisten Softwareentwickler oder auch
Gestalter mit diesen Technologien vertraut, zudem beschäftigen sich
Laien sehr oft auch mit Webtechnologie, allein weil sie z.B. eine kleine
eigene Webseite erstellen wollen.

Bündelung von verschiedenen Webdiensten wird ermöglicht, so könnte man sich
vorstellen bestimmte Informationen über z.B. die Wolfram|Alpha API
anzufordern oder verarbeiten zu lassen oder wenn es nur eine Weiterleitung
als Metainformation ist.
Zusammengefasst:

\begin{itemize}
  \item flexible Layouts,
  \item hochwertige Schriftarten verfügbar,
  \item leichte Erweiterbarkeit,
  \item viele Technologien,
  \item Standardisiert und stark entwickelt → Zukunftssicherheit,
  \item ubiqutär eingesetzt → keine zusätzlichen Softwareinstallationen,
  \item bekannte Materie für viele Menschen,
  \item Verwendung verschiedener Dienste über Web-APIs.
\end{itemize}

\subsection{Interaktive Fähigkeiten}

Webtechnologien haben nicht nur die o.g. Fähigkeiten, sondern gerade
HTML5 glänzt mit seinen interaktiven Fähigkeiten, welche einem relativ
statischen Dokument zusätzliche Möglichkeiten bietet Dinge zu
visualisieren und dennoch druckbar zu halten.

Die neuen visuellen Fähigkeiten sind nicht der einzge Vorteil der sich ergibt,
sondern der Benutzer kann auch viel mehr mit dem Dokument interagieren
und mit anderen Personen die in irgendeiner Form an dem Dokument beteiligt
sind, z.B. durch Kommentare bzw. Diskussionen direkt im Dokument.

Auch können Daten dynamisch nachgeladen werden, z.B. ein Plot der sich
im Dokument bei Änderungen oder neuen Daten automatisch aktualisieren kann.

Anreicherung mit Metadaten ist auch möglich, die Daten können dem Dokument
implizit mitgeliefert werden, so dass Metainformationen über ein Maus-Hover
eingeblendet werden oder eine extra Informationsbox aufgeht oder einfache
Querverweise, z.B. zur komfortablen Referenzierung auf Quellen,
auf andere Dokumente oder Internetadressen.

\subsection{Automatisierungs- und Programmierungsfähigkeiten}

Als Schnittstelle zur Erstellung eines Dokuments soll Programmcode
dienen, ähnlich wie es bei LaTeX und Derivaten der Fall ist, nur
dass eben eine vollwertige Programmiersprache zum Einsatz kommt, diese
dann viele neue Türen öffnet, um die Dokumentengenerierung weiter
zu automatisieren oder mit mehr dynamischen bzw. generierten Inhalten
füllen kann.

\emph{Die Plattform dient zur Kooperation zwischen verschiedensten
Funktionen, Programmen und Diensten.}

Der Programmcode übernimmt auch Dinge wie die automatische Durchnummerierung
von Kapiteln, oder Auflösung von Querreferenzen auf Kapitel, Bilder oder
Literatur. Dadruch, dass es Programmcode ist, kann jedes beliebige Verhalten
hinzugefügt oder modifiziert werden, so dass es auf die Anforderungen
des gewünschten, eventuell speziellen neuartigen, Dokumententypus
zugeschnitten werden kann.

\paragraph{Szenario 0:} Makrdown, Txt, Html (doc?, tex?) \ldots
Bsp. Komplexe und halbautomatische Tabellen.

% TODO schreiben, Word doc integrieren, xls/csv Datei als input als tabelle
% tabelle gleich als datenstruktur nutzbar zur weiterverarbeitung, rechnung,
% anreicherung
% --> alle möglichen technologien kooperieren lassen und auf einen nenner bringen!
% ---> bessere kolloboration (jeder kann sein gewohntes tool nutzen)

\paragraph{Szenario 1:} Immer wenn das Dokument „gebaut“ wird, kann z.B.
automatisch ein Skript angeworfen werden, welches einen Plot als Bild
anfertigt. Dieses Bild kann dann direkt in das resultierende Dokument ohne
Umwege eingebaut werden. Vorteil: Der Dokumenten-Ersteller muss also nicht jedes
mal wenn er den Plot verändert, ihn nochmals manuell anfertigen und
die Bilddatei ins richtige Verzeichnis schieben. → Weniger Handarbeit,
die Dokumenten-Erstellungsumgebung muss nicht verlassen werden.

\paragraph{Szenario 2:} Durch eine vollwertige Programmiersprache hat
der Dokumenten-Ersteller Zugriff auf verschiedene Programm-Bibliotheken,
um z.B. Daten aus einer Datenbank oder dem Dateisystem zu holen, diese
zu verarbeiten, aufzubereiten und im Dokument darzustellen, beispielsweise
als Plot. → Zugriff auf immer aktuelle Daten, mit jeder Dokumenten-Erzeugung.

% TODO Excel Popup (mit VB Script?)  <-> Einbettung im verschiedene Umgebungen.

\paragraph{Szenario 3:} Ermöglicht Konvertierung von verschiedenen
Konventionen auf einen gemeinsamen Nenner, z.B. bei chemischie Formeln
gibt es viele Konventionen der Dateiformale (populär sind SMILES, Molfiles oder
IUPAC-Namen.) Eine Bibliothek könnte all diese Formate annehmen und
für die Webansicht konform machen.

Zudem könnte eine implizite Anreicherung der Informationen vorgenommen werden,
so dass der Benutzer im Programmcode lediglich „Zeichne Strukturformel
für Coffein“ angibt, aber die Bibliothek noch weitere Hintergrundinformationen
zu Coffein aus anderen Quellen (z.B. Wikipedia, Wolfram|Alpha, \ldots) bezieht.

\paragraph{Szenario 4:} Dokumentation
und lauffähiger Code vereint, so dass das Dokument quasi auch schon selbst
die Problemlösung errechnen kann.

Beispiel: Das Dokument beschreibt eine Simulation, zu einem gestellten
Problem. In dem Dokument selbst ist ein Algorithmus beschrieben, der
für die Simulation eingesetzt wird. Aber dadruch, dass das Dokument selbst
in einer vollständigen Programmiersprache geschrieben wird, kann dieser
Algorithmus direkt während der Dokumenten-Generierung ausgeführt werden
und mit Daten aus einer Datenbank gefüttert werden und entsprechende Plots
im resultierenden Dokument darstellen. Also ist der Code, der im Dokument
beschrieben ist auch gleichzeitg der funktionsfähige und ausführbare
Algorithmus.

\paragraph{Szenario 5:} Man könnte sogar so weit gehen, und statt eine einfache
zu HTML-Seite zu generieren einen Webservice starten, der dem Dokument noch
mehr dynamische Fähigkkeiten ermöglicht, z.B. durch bidirektionale
Kommunikation zwischen Webbrowser-Client (Dokumenten-Betrachter) und
Dokumenten-Server, welche eine Interaktion zwischen Benutzer und Dokument
bzw. anderen Benutzern des Dokuments ermöglicht.

\subsection{Zusammengefasst / Was getan werden muss / Was dafür gebraut wird}
% TODO fertig machen
\begin{itemize}
  \item Vermischung von statischen und automatisch generierten
        Doku\-menten-\-Be\-stand\-teilen,
  \item Datenaufbereitung quasi zur Laufzeit der Doku\-menten-\-Er\-stell\-ung,
  \item Strukturierungsmöglichkeiten durch den Quellcode, in Pakete, Klassen
        → Objekt-Orientierung,
  \item Webtechnologie ermöglicht dynamische Inhalte,
  \item Webtechnologie ist reaktiv (z.B. auf den Benutzer, Inhalte nachladen),
  \item Gute Kolloberationsmöglichkeiten, Verwaltungsmöglichkeiten,
        da Quellcode
  \item Verknüpfung verschiedener Technologien (Datenbanken, Dateisystem,
        Interpozesskommunikation, etc.),
  \item Sehr flexible Gestaltung des Dokuments, da Webtechnologie möglich,
  \item Webtechnologie ermöglicht Rückkanal, z.B. kollaborierende Benutzer
        können Kommentare schreiben, oder mehr. (Richtung Google Docs.),
  \item Viele Erweiterungsmöglichkeiten, geg. durch Programmiersprache und
        Webtechnologie,
  \item Webtechnologie hat eine sichere Zukunft und ist standardisiert.
\end{itemize}

Es muss also ein kleines JavaScript-Framework entwickelt werden, welches die
Aufgabe der Darstellung des Dokuments übernimmt. Die Zielachitektur.

Zudem braucht es noch ein Programm bzw. Programmiersprache, welches diese
Zielarchitektur füttern kann. Dieses Programm soll Aufgaben wie z.B.
Kapitel-Nummerierung automatisch abwickeln. Weiterhin muss es auch ein
wohlgeformte Schnittstelle zum Benutzer liefern. Diese Kriterien führen
dazu, dass die Entwicklung einer Domänen-Spezifischen Programmiersprache,
kurz DSL, sehr sinnvoll ist.

\section{Eine Wendung}

Als ich soweit an der Thematik getüftelt hatte und die Machbarkeit als
Bachelor-Thesis erkannte trug ich den Vorschlag am ISE vor, jedoch sind
diese kein Informatik Institut und sahen sich nicht in der Lage die Arbeit
zu betreuen, wenngleich sie die Idee sehr nützlich fanden. So wurde mir
nahe gelegt, dass ich bei einem anderen Fraunhofer Institut anklopfen könne.

Ich habe das Fraunhofer SCAI angeschrieben, und Dr. Marc Zimmermann fand
die Idee spannend und auch passend für deren Themengebiet. Sie bereiten u.a.
Patente auf, indem sie eine Patent-PDF-Datei mit Hilfe ihres Java-Framework
zerlegen und die so erhaltenen Daten ggf. mit zusätzlichen Informationen
anreichern. Die Idee von mir hat ihnen sehr zugesagt, da sie noch eine
Möglichkeit suchten, die die aufbereiteten Patente mit Webtechnologie
darzustellen kann bzw. auszuliefern.

\section{Zum Dokument}

In diesem Dokument versuche ich nach Möglichkeit \emph{vollständig}
deutsche Sprache anzuwenden, also nur dort wo es unumgänglich ist
Anglizismen zu verwenden. Das gilt auch gerade für Fachsprache, sofern es
passende deutsche Übersetzungen gibt.


\chapter{Frontend: Entwicklung eines Frameworks}

Übersicht über das Kapitel.

\section{Idee und Aufgaben}

\section{Architektur}

\section{Implementierung}

Vielleicht die Details bzw. Doku-Details in die Appendix auslagern?

\section{Essenz}


\chapter{Backend: Domain-Specific Language}

Übersicht über das Kapitel.

\section{Idee und Aufgaben}

\section{Vergleich}

\subsection{Entscheidung}

\section{Architektur}

\section{Implementierung}

\section{Essenz}


\chapter{Lösungsweg}

Übersicht über das Kapitel.

\section{Vergleichsmatrix}

\begin{longtable}{|l|l|l|l|l}

  \hline
  Nr. & Fähigkeit & Xtext (externe DSL) & Scala (interne DSL) \\ \hline \hline
  \endfirsthead

  \hline
  Nr. & Fähigkeit & Xtext (externe DSL) & Scala (interne DSL) \\ \hline
  \endhead

  & Foo
  & Bla
  & Blub
  \\
  \cline{3-4}
  & & \multicolumn{1}{c|}{ 1 } & \multicolumn{1}{c|}{ 2 } \\ \hline

\end{longtable}

\chapter{Architektur}

Architektureller Aufbau der Softwaresysteme.

\section{Vorwärtsverweis „Forward Reference“}

Dieses Problem tritt auf, wenn z.B. in einem Text auf eine Abbildung
verweisen wird, welche innerhalb des Programmflusses erst später verfügbar
wird.

\begin{lstlisting}
... auf Abbildung {Reference} ist zu sehen ...

Reference = new Figure(...)
\end{lstlisting}

Hier wird also auf \lstinline|Reference| bereits zugegriffen,
bevor sie überhaupt existiert. Erschwerend kommt hinzu, dass sich die
Reihenfolge der Entitäten (Texte, Bilder, etc.) nicht verändert werden darf.
Die Abbilung soll also an der Position im Dokument erscheinen, an der sie
auch im Dokument-Quellcode geschrieben wurde, da es sich hier um eine DSL
handelt, die sich möglichst nahme am eigentlichen Dokument orientiert.

% Note:
% Von oben nach unten, wir die Lösung immer unflexibler, unsicherer und
% aufwendiger in der Implementierung. Wobei Request-Queue und Pattern-Matching
% etwas zusammengehören.


\subsection{Closure}

Insbesondere funktionale Programmiersprachen wie Scala haben die
Möglichkeit Closures zu bilden, d.h. der Teile Geltungsbereich (Scope)
der äußeren Funktion  kann von der inneren Funktion beibehalten werden,
auch wenn der Geltungsbereich der äußeren Funktion bereits verwirkt ist.
% TODO(oder in appendix?)

\begin{lstlisting}
def outer_func = {
  val v = 15
  (x: Int) => x + v  // lambda function
}
\end{lstlisting}

Auf \lstinline|v| kann noch über die Lambda-Funktion zugegriffen werden,
selbst wenn \lstinline|outer_func| nicht mehr exisiert. Teile des
\lstinline|outer_func|-Geltungsbereichts werden quasi mitgezogen.

Genau mit dieser Technik kann man den Vorwärtsverweis in den Griff bekommen.
Es wird der äußere Geltungsbereich eines Objects nach innen gezogen,
um dort später wenn alle Entitäten bekannt sind und die Referenz aufgelöst
wurde darauf zuzugreifen.

\begin{lstlisting}
object O {
  val text = () => s"… auf Abbildung $reference ist zu sehen …"
  val reference = 3
}

O.text()  // wenn reference vorhanden
\end{lstlisting}

Hier wird zudem die Eigenschaft des \lstinline|object| ausgenutzt,
dass die im \lstinline|object| genannten Variablen immer schon vom Compiler
zumindest mit einer \lstinline|null|-Referenz exisieren, aber die Reihenfolge
der eigentlichen Instanziierung wird nicht verändert. Durch das
\lstinline|lazy|-Keyword von Scala, würde die Reihenfolge modifiziert werden
und ist dadurch nicht verwendbar.

Nachteil hier ist, dass der Domänen-Benutzer innerhalb der DSL diese
\lstinline|() => s""|-Magie schreiben müsste -- was zu Verwirrung und
Unverständnis führen würde.


\subsubsection{Verbesserung für den Domänen-Benutzer}

Ideal wäre also nun eine Lösung inder der Domänen-Benutzer keine
Aufmerksamkeit auf die Closure-Magie verschwenden muss.

Wie bereits in einem vorherigen Kapitel % TODO(oder in appendix?)
erwähnt wird auf den ab Scala 2.10 verfügbaren
\lstinline|StringContext| (\lstinline|s"…"|) gesetzt.
Dieser lässt sich so erweitern, dass
sich die Closure-Magie verstecken lässt und somit in die API
gezogen wird und der Domänen-Benutzer davon gar nichts mitbekommt.

\begin{lstlisting}
implicit def byname_to_noarg[A](a: => A) = () => a

case class StringContext(parts: String*) {
  def $ (args: (() => Any)*) = () => {
    val unpacked_args = args.map(a => a())
    scala.StringContext(parts: _*).s(unpacked_args: _*)
  }
}

object O {
  val text = $"… auf Abbildung $reference ist zu sehen …"
  val reference = 3
}

O.text()  // wenn reference vorhanden
\end{lstlisting}

\lstinline|byname_to_noarg| ist eine implizite Konvertierung von einem
beliebigen Typ \lstinline|A| mit Call-by-Name
% TODO(fußzeile, appendix "benutzte scala technologien"?)
\lstinline|a: => A| in eine Lambda-Funktion \lstinline|() => a|.

\lstinline|def $| fügt die Möglichkeit hinzu \lstinline|$"…"| als individuell angepassten \lstinline|StringContext| zu verwenden. Es wird eine variable 
Argumentenliste mit den implizit zu Call-by-Name konvertierten Argumenten
aus einem beliebigen \lstinline|$"…"|-String übergeben und in ein Closure gepackt,
welches erst dann ausgeführt wird, wenn die Referenzen auch tatsächlich
vorhanden sind.

% http://stackoverflow.com/questions/13307418/scala-variable-argument-list-with-call-by-name-possible

% http://stackoverflow.com/questions/13270906/cast-scala-string-to-stringcontext-and-virtually-forward-references

\subsection{Eval}

\subsection{Datenstrukturbasierend}

\subsubsection{Request-Queue}

Nur Idee!

\subsubsection{Pattern-Matching}

Nur Idee!

\subsubsection{„place here“}

Praktikabel mit Xtext-Generator. Prototypisierung ausnutzen (C-Header).


\addcontentsline{toc}{chapter}{Literaturverzeichnis}
\bibliographystyle{geralpha}  % alphadin
\bibliography{bib/references}

%\listoffigures
%\listoftables

\end{document}

