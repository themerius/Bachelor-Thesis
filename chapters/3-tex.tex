\subsection{\TeX, eine DSL für Dokumente}

\TeX~ ist eine weit verbreitete Programmiersprache die speziell für die
Aufgaben die ein Textsatzsystem benötigt erschaffen wurde. Daher kann man
\TeX~ auch als eine DSL ansehen. In diesem Abschnitt wird etwas auf
\TeX~ eingegangen, da insbesondere diese Programmiersprache als Inspiration
bzw. Vorbild dient.


\subsubsection{Herkunft und Grundlagen}

Der Name hat seinen Ursprung aus dem griechischen $\tau\epsilon\chi$,
welches auch die Wurzel für das englische Wort \emph{technology} ist.
$\tau\epsilon\chi$ bedeutet also Technologie, aber auch Kunst.
(\cite{tex-a}, Kapitel 1, Seite 1)

\TeX~ist ein Textsatzsystem, welches für die Erstellung
von qualitativ hochwertigen Büchern ausgelegt ist, mit einem starken Fokus auf
Mathematik.

Es gibt etwa 300 \TeX~Kontrollsequenzen, sog. „Primitive“ welche das
Low-Level TeX bilden. Diese Primitiven sind atomisch und werden nicht weiter
in kleinere Funktionen zerlegt.
Zudem kommen noch etwa 600 weitere, aus Primivien zusammengesetzte,
Kontrollsequenzen „plain \TeX“ dazu,
die zusammen mit den Primitiven das Standard-\TeX~bilden.
(\cite{tex-a}, Kapitel 3, Seite 9--11)

\TeX~ hat eine \emph{REPL}\footnote{REPL steht für read–eval–print loop,
eine interaktive Programmierumgebung. Scala besitzt auch eine REPL.}
in welcher interaktive Programmiersitzungen abgehalten werden können,
wie z.B.:

\paragraph{Primitiv}

\begin{verbatim}
**\show\input
> \input=\input.
\end{verbatim}

\paragraph{Zusammengesetzte Kontrollsequenz, ein Makro}

\begin{verbatim}
**\show\TeX
> \TeX=macro:
->T\kern -.1667em\lower .5ex\hbox {E}\kern -.125emX.
\end{verbatim}


\subsubsection{Fähigkeiten als Programmiersprache}


\subsubsection{Warum \TeX~ so aussieht, wie es aussieht}


\subsubsection{Was \TeX~ generiert}


\paragraph{Produktionsformat versus Auslieferungsformat}

\TeX~ stellt ein Konzept besonders klar dar: die \emph{Unterscheidung
zwischen Produktionsformat und Auslieferungsformat.} Diese Unterscheidung
wird von vielen Menschen nicht verstanden oder wahrgenommen; daher kommt
es immer mal wieder vor, dass Microsoft Word Dateien im E-Mail Anhang zu
finden sind.

Bei \TeX~ ist das Produktionsformat die Quellcode-Dateien und in neuerer
Zeit ist das Auslieferungsformat eine gesetzte PDF-Datei, welche auf jeder
Plattform gleich angezeigt wird und auch als Archivierungsformat tauglich
ist.


\subsubsection{The Good, the Bad and the Ugly}
