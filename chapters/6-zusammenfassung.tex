\chapter{Zusammenfassung}

Im Verlaufe dieser Arbeit konnte gezeigt werden, dass DSLs eine sehr
spannende Technologie sind, um das Problemfeld der automatischen
Dokumentengenerierung bzw. Textsatz anzugehen. % Der Klassiker TeX...

Dazu wurden zwei DSL-Technologien in Kapitel \ref{sec-vergleich} verglichen:
Das Xtext-Framework zur Erstellung \emph{externer DSLs} und
die Scala-Programmiersprache zur Erstellung \emph{interner DSLs}.

Erstellung qualitativ hochwertiger Dokumente war,
ist und bleibt eine Herausforderung.
Das hier erforschte Textsatzsystem versucht die ubiquitäre und hoch entwickelte
Webtechnologie für die Darstellung klassischer Dokumente bzw. Printdokumente
zu verwenden.
Mit Vorteilen ausgerüstet wie hoher gestalterischer Freiheit,
interaktiven Inhalten und aktuellster Technik ist Webtechnologie
mittlerweile\footnote{
HTML5-Technologien haben eine immense Qualitätssteigerung einher gebracht.}
eine sehr gute Basis zur Darstellung von hochwertigen Dokumenten.

Mithilfe der internen Scala-DSL wird dem Dokumentenersteller eine intuitive
Methode bereitgestellt, Dokumente auf Basis der Webtechnologie zu generieren
und dabei die Möglichkeit zu haben, innerhalb seines Dokumenten-DSL-Skripts
die Mächtigkeit von Scala zu nutzen---was wiederum ausschweifende
Automatisierungsfähigkeiten ermöglicht.

Dieses Projekt wird bereits vom Fraunhofer SCAI für die Aufbereitung von
chemischen Patentschriften eingesetzt.

\section{Fazit über die Ergebnisse}

\paragraph{Vergleich.}
Der Vergleich der zwei DSL-Technologien ist
in der Vergleichsmatrix in Kapitel \ref{sec-vergleichsmatrix} aufgeführt,
darin werden 15 Fähigkeiten beschrieben und jeweils Xtext mit Scala
gegenübergestellt.

Mit Scala als Gewinner des Vergleichs wird der DSL-Teil Textsatzsystems, als
interne DSL, implementiert. Da eine interne Scala-DSL einer Softwarebibliothek
entspricht, sind somit alle Scala-Programmierfähigkeiten im DSL-Skript nutzbar;
Scala eine starke Sprachinfrastruktur hat und z.B. eine
ausgereifte Code-Strukturierung
mitliefert; Scala kann es einem DSL-Benutzer ermöglichen, selbst Dokumententemplates
zu schreiben; die Scala-DSL ist sehr gut in andere Kontexte einbettbar und ist
dank Scalas Skalierbarkeit auch für die Zukunft gut gerüstet; Elegante Lösung
des Vorwärtsverweisproblems---siehe Kapitel \ref{sec-forwardreference}.
Mehr dazu in Kapitel \ref{sec-auswertung}.

Dieser Vergleich auch allgemeinen für \emph{externe versus
interne DSLs} gelten, wenn die Xtext- bzw. Scala-spezifischer Besonderheiten
abgezogen werden. Da das Ergebnis
auf den Anwendungsfall „Textsatzsystem“ getrimmt ist muss ggf. eine Anpassung
vorgenommen werden, falls die Ergebnisse für andere Projekte adaptiert werden sollen.

\paragraph{Implementierung: Zielarchitektur (Darstellung).}
Normalerweise trennen Webseiten den Inhalt nicht noch in weitere Seiten auf,
wie es bei z.B. Printdokumenten der Fall ist.
Ein solches Verhalten wurde mittels JavaScript innerhalb dieser Arbeit
ermöglicht.
Die Zielarchitektur kann also eine Webseite wie ein Printdokument aussehen
lassen und bietet mittels Webtechnologie ein reichhaltiges Angebot an
Templating und Gestaltungsfähigkeiten.

Während der Entwicklung entstanden u.a. diese Dokumententemplates: (1)
der Fraunhofer Bericht und (2) die Patentschrift des europäischen
Patentregisters. Erreicht wurde zudem:

\erreichtZielarchi

\paragraph{Implementierung: DSL und Generator.} Es konnte gezeigt werden,
dass eine intuitive DSL zur Dokumentengenerierung, inspiriert von \TeX,
auch mit Scala vernünftig umsetzbar ist---ein ausführliches Beispiel
für ein DSL-Skript ist im Anhang \ref{sec-api-resultat} und
nähere Erläuterungen dazu in Kapitel \ref{sec-api-design}.

Das Problem des Vorwärtsverweises, welcher in Dokumenten oft auftritt,
konnte zudem elegant mit Scala-Abstraktionen und eines Closures gelöst werden.
Erklärung in Kapitel \ref{sec-forwardreference}. Erreicht wurde:

\erreichtDSL

\section{Ausblick}

Dieses Projekt in viele Richtungen weiterentwickelt werden werden,
wie in Abschnitt \ref{sec-idee} erläutert---hier einige Beispiele:

\begin{itemize}
  \item Die DSL kann erweitert werden,
        um eine Syntax für z.B. Tabellen zu schaffen.
  \item Weitere Logik Klassen die z.B. ermöglichen verschiedene Formate wie
        Markdown zu importieren und intern in Entitäten umzuwandeln.
  \item Weiterentwicklung der Webtechnologie-Darstellung, insbesondere
        sollte noch ein Algorithmus der Entitäten, wenn nötig,
        weiter aufteilt um so eine möglichst hohe Auslastung der Seiten
        zu ermöglichen.
  \item Kommunikation mit einem Webserver (via JavaScript) für mehr Interaktion,
        z.B. für Benutzerkommentare im Dokument oder Daten die sich automatisch
        aktualisieren.
  \item Neuartige Templates, die z.B. ein Dokument leicht als Mindmap
        darstellen lassen oder aus dem Dokument eine Präsentation erstellen.
  \item Die Templates brauchen noch striktere Paradigmen, falls mehr
        Templates gewünscht werden, so dass sie leicht austauschbar bleiben.
        Beispielsweise jedes Template hat ein vier Spaltenlayout und n definierte
        Zeilen.
\end{itemize}