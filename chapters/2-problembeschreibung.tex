\chapter{Problembeschreibung → Aufgabenstellung}

In diesem Kapitel soll eine umfassende Beschreibung des Problems erfolgen,
aus welcher sich wiederum die Aufgabenstellung ergibt, welche im
späteren Verlauf implementiert wird.

Den Anfang macht die Zielachitektur, welche primär aus Webtechnologien
besteht und somit zur Darstellung des Dokumentes dient. Siehe
Abschnitt \ref{sec-zielarchitektur}.

Die Zielarchitektur soll von einem Benutzer ohne großen Aufwand
generierbar sein und soll dabei die Möglichkeit haben, bestimmte
Aufgaben automatisch zu erledigen, wie z.B. das Zusammenstellen
eines Inhaltsverzeichnisses. Siehe Abschnitt \ref{sec-dsl}.

Damit aus der Schnittstelle die dem Benutzer gegebenen ist die Zielarchitektur
generiert werden kann, muss es eine geeignete Brücke bzw. Verbindung
geben, diese wird in Abschnitt \ref{sec-verbindung} näher beschrieben.

% TODO Verweise auf die Architektur!

\section{Zielarchitektur}\label{sec-zielarchitektur}

Die primäre Aufgabe der Zielarchitektur ist es ein Dokument in
einem Webbrowser darzustellen. Das Dokument ist aber keine gewöhnliche
Webseite, sondern orientiert sich hautsächlich an akademischen
Dokumenten wie z.B. Fachbüchern. Bücher bestehen auch einzelnen Seiten,
diese sind in Webseiten in ihrer Ursprünglichen Form nicht vorgesehen,
eine stellt lange Dokumente stufenlos, also ohne Seiten dar.
Daher ist eine Aufgabe der Zielarchitektur eine \emph{Abstraktion
über Seiten} für Webtechnologie zu bieten.

Die eigentliche Problematik ist hierbei, dass der Webbrowser fließenden
Text nicht selbst auf definierte Seiten umbrechen kann. Für den Browser
gibt es nur eine Seite die quasi einem langen Textschlauch entspricht,
wenngleich er durch HTML struktuirert und
CSS-Eigenschaften gestaltet werden kann.

Damit verbunden ist die Problematik der
\emph{Zuordnung der einzelnen Dokumenten-Entitäten zu den Seiten},
dabei kann es dazu kommen dass eine Entität zwischen zwei Seiten
\emph{überlappt} und diese sollte, wenn möglich, passend aufgeteilt werden.

Dadurch dass die Zielarchitektur sich um die Erstellung der Seiten und
die Zuteilung der Entitäten kümmern muss, ist es vollkommen natürlich,
dass sich diese auch der \emph{Nummerierung der Seiten} annehmen muss.
Davon hängt auch dirket die Funktion, die einzelnen Punkte des
Inhaltsverzeichnisses mit der entsprechden Seitennummer zu versehen, ab.

Zudem sollte es möglich sein, dass es \emph{verschiedene Arten von Seiten}
gibt, z.B. Deckblatt, normale vertikale Seite und horizontale Seiten für
große Tabellen oder Abbildungen. Oder auch dass verschiedene Papierformate
wie DIN-A4 oder US-Letter zur Auswahl stehen.

Die Einführung von \emph{Dokumenten-Arealen} als eine weitere hilfreiche
Abstraktion die zur Gliederung eines - insbesondere
akademischen - Dokuments dient, ist mit Sicherheit nicht verkehrt.
Damit soll es möglich sein, Dinge wie beispielsweise
Deckblatt, Inhaltsverzeichnis, eigentliches Dokument
und Literaturverzeichnis etc. von einander zu separieren für eine
flexiblere Konfiguration, z.B. um die Areale in der Anordnung zu
verändern, oder Dinge wie Nummerierungen zu modifizieren, beispielsweise
wird gerne der Anfang eines Dokuments mit römischen Ziffern durchnummeriert
und das eigentliche Dokument wird jedoch mit
arabischen Ziffern durchnummeriert.

% Seitennummerierung
% Inhaltsverzeichnis, Seite zuordnen
% Dokumentenareale
% Vorteile die sich durch die HTML-Nutzung ergeben? Welches Kapitel?

\paragraph{Anforderungen} auf einen Blick zusammengefasst:

\begin{itemize}
  \item Abstraktion für Seiten,
  \item Zuordnung der Entitäten auf Seiten,
  \item Behandlung überlappender Entitäten,
  \item Ermöglichung verschiedener Seiten-Arten,
  \item Durchnummerierung der Seiten,
  \item Seitenzugehörigkeit von Entitäten bestimmbar,
  \item Dokumenten-Areale zur Strukturierung.
\end{itemize}


\section{Domain-Specific Language}\label{sec-dsl}

% TODO wie es zum Vergleich kommt
% TODO was ist eine DSL? Definition DSL!

\section{Verbindung zwischen DSL und Ziel}\label{sec-verbindung}


