\chapter{Lösungsweg}

Übersicht über das Kapitel.

\section{Vergleichsmatrix}

\begin{landscape}
\begin{longtable}{|p{0.5cm}|p{4.5cm}|p{6.5cm}|p{6.5cm}|}

  \hline
  Nr. & Fähigkeit & Xtext (externe DSL) & Scala (interne DSL) \\ \hline \hline
  \endfirsthead

  \hline
  Nr. & Fähigkeit & Xtext (externe DSL) & Scala (interne DSL) \\ \hline
  \endhead

  & DSL als Library bzw. Deployment-möglichkeiten
  & Ist eine in sich mehr oder weniger geschlossene Struktur.
  & Interne DSL ist eine ganz normale Scala Library.
  \\
  \cline{3-4}
  & & \multicolumn{1}{c|}{ 1 } & \multicolumn{1}{c|}{ 2 } \\ \hline

  & Sprach-Infrastruktur
  & Xtext generiert automatisch ein speziell angepasstes Eclipse Plugin.
  & Alles wird mitgeliefert, wie z.B. Compiler, Built-Tools, REPL.
    Breite Unterstützung von vielen Editoren.
  \\
  \cline{3-4}
  & & \multicolumn{1}{c|}{ 1 } & \multicolumn{1}{c|}{ 2 } \\ \hline

  & Strukturierungsfähigkeit des Codes
  & Muss alles selbst gebaut werden. Vorteil: Es muss nur das nötigste
    umgesetzt werden.
  & Sämtliche Infrastruktur vorhanden. (Packages, Kontrollstrukturen,
    Build-Tools, ...)
  \\
  \cline{3-4}
  & & \multicolumn{1}{c|}{ 1 } & \multicolumn{1}{c|}{ 2 } \\ \hline

  & DSL mit General Purpose mischbar
  & Hat viele Hürden, um eine DSL mehr Allgemeingültigkeit zu verpassen
  & Alle Scala-Fähigkeiten nativ nutzbar, da die DSL eine normale Library ist.
  \\
  \cline{3-4}
  & & \multicolumn{1}{c|}{ 1 } & \multicolumn{1}{c|}{ 2 } \\ \hline

  & Toolsets (für DSL Gestaltung)
  & Komplette und entsprechend angepasste Eclipse Entwicklungsumgebung.
  & Die Sprace selbst, sonst keine Hilfen.
  \\
  \cline{3-4}
  & & \multicolumn{1}{c|}{ 1 } & \multicolumn{1}{c|}{ 2 } \\ \hline

  & Erweiterbarkeit durch Entwickler
  & Grammatik, Tests und Generator kann nach belieben wachsen, u.a.
    Unterstützung durch Eclipse.
  & Der Aufwand liegt bei der Entwicklung einer Library. Jedoch müssen
    Testumgebungen etc. selbst eingerichtet werden.
  \\
  \cline{3-4}
  & & \multicolumn{1}{c|}{ 1 } & \multicolumn{1}{c|}{ 2 } \\ \hline

  & Erweiterbarkeit durch Domain User/Community (z.B. für eigene Templates)
  & Es würde von dem Domain User verlangt werden BNF-Notation zu können,
    Xtend und er wäre auf Eclipse gezwungen.
  & Einfache Scala Kenntnisse plus eine kleine Anleitung sollten ausreichen,
    die Bindings zu erstellen.
  \\
  \cline{3-4}
  & & \multicolumn{1}{c|}{ 1 } & \multicolumn{1}{c|}{ 2 } \\ \hline

  & Wiederverwendbarkeit bzw. Kombination mit Vorhandenem
  & Nur eingeschränkt, jedoch sind Grammatik Mixins möglich.
  & Sehr gut, da Library und mit Scalas Typ- und Vererbungssystem kann nach
    gewohnter Manier kombiniert und erweitert werden.
  \\
  \cline{3-4}
  & & \multicolumn{1}{c|}{ 1 } & \multicolumn{1}{c|}{ 2 } \\ \hline

  & Grammatikalische Gestaltung der DSL
  & Komplett frei und flexibel, da in BNF-Regeln definiert.
  & Eingeschränkt, man bleibt an Scala's Beschränkungen gebunden.
  \\
  \cline{3-4}
  & & \multicolumn{1}{c|}{ 1 } & \multicolumn{1}{c|}{ 2 } \\ \hline

  & Generator: Zielplatform
  & Ohne Umwege kann jede Sprache oder Markup aus dem DSL-Modell durch eine
    Template-Engine generiert werden, das Eclipse-Plugin stellt sofort das
    Generat bereit. Jedoch kann nativer Code nicht direkt auf Xtext laufen,
    es muss also ggf. noch ein externer Build o.ä. angestossen werden.
  & Die DSL selbst kann direkt ein lauffähigkes Programm sein. Andere Ziele,
    z.B. andere Programmier-Sprachen oder Markup-Sprachen müssen einen Umweg
    über eine Template-Engine nehmen, allerdings steht hier ein Eclipse-Plugin
    bereit, welches direkt nach jeder Änderung das Generat bereitstellt; das
    Verfahren hierzu muss selbst entwickelt werden (das kann ein Vor- oder
    auch ein Nachteil sein.)
  \\
  \cline{3-4}
  & & \multicolumn{1}{c|}{ 1 } & \multicolumn{1}{c|}{ 2 } \\ \hline

  & Generator: Möglichkeiten der Template-Engine
  & Xtend eine speziell angepasste DSL-Generator-Template-Engine.
    Die BNF-Grammatik wird transparent in Java- bzw. Xtend-Klassen übersetzt,
    mit denen das Ziel über das Template generiert werden kann.
  & 1. Freie Wahl, z.B. einface Multiline-Strings, Scala XML oder Scalate;
    wie aus der internen DSL das Ziel generiert wird, benötigt in der Regel
    einen Zwischenschritt (Bindings), welcher programmiert werden muss.
    Scala kann jedoch ggf. das Generat als Unterprogramm ausführen.
    2. Die interne DSL ist selbst lauffähig.
  \\
  \cline{3-4}
  & & \multicolumn{1}{c|}{ 1 } & \multicolumn{1}{c|}{ 2 } \\ \hline

  & Entwicklungsaufwand (u.a. Zeit, Einarbeitung)
  & Wenn BNF-Kenntnisse (th. Informatik) vorhanden sind, leichte Einarbeitung.
    Die Tools nehmen die harte Arbeit ab. Es gibt schon standardisierte
    Vorgehensweisen, z.B. wie der Generator gebaut wird.
  & Wenn Scala-Kenntnisse vorhanden, ist es mehr oder weniger die Entwicklung
    einer Library. Wie man den Generator baut, muss allerdings überlegt werden.
  \\
  \cline{3-4}
  & & \multicolumn{1}{c|}{ 1 } & \multicolumn{1}{c|}{ 2 } \\ \hline

  & Software-Lebenszyklus und Wartbarkeit
  & Dank IDE und einer schon eingerichteten Testumgebung, also sehr gut.
  & Man hat alle Möglichkeiten, die die Scala-Welt bietet, also sehr gut.
    Jedoch ist Handarbeit nötig.
  \\
  \cline{3-4}
  & & \multicolumn{1}{c|}{ 1 } & \multicolumn{1}{c|}{ 2 } \\ \hline

  & Skalierbarkeit, Umbebungs-Einlagerung
  & Kommt auf das Generat an. Man ist und bleibt an Eclipse gebunden.
  & Scala selbst ist in alle Richtungen (Größe, Nebenläufigkeit) sehr gut
    skalierbar.
  \\
  \cline{3-4}
  & & \multicolumn{1}{c|}{ 1 } & \multicolumn{1}{c|}{ 2 } \\ \hline

  & Umbebungs-Einlagerung
  & De facto Eclipse-Bindung, aber mit individuell angepasstem Eclipse-Plugin,
    welches sich mit dem Projektverlauf automatisch mit anpassen kann.
    Wenn das Ziel ein Arbeitsplatz-Front-End ist, sehr vorteilhaft -- sofern
    Eclipse eingesetzt werden will.
  & Kann gut in alle möglichen Szenarien eingebettet werden, Benutzung
    innerhalb eines Frameworks möglich, oder einsatz als Bibliothek,
    Stand-Alone oder in einer Entwicklungsumgebung wie Eclipse denkbar.
    Allerdings ist Xtext in Eclipse besser eingebunden, da speziell angepasst.
  \\
  \cline{3-4}
  & & \multicolumn{1}{c|}{ 1 } & \multicolumn{1}{c|}{ 2 } \\ \hline

\end{longtable}
\end{landscape}
