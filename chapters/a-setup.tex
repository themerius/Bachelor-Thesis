\chapter{Einrichtung und Betrieb}

GitHub Repositoria:

\begin{itemize}
  \item Zielarchitetkur: \url{https://github.com/themerius/ScalTeX-templates}
  \item DSL: \url{https://github.com/themerius/ScalTeX-DSL}
\end{itemize}

\section{Voraussetzungen}

\begin{itemize}
  \item Java Virtual Machine bzw. OpenJDK in Version 6 oder 7,
  \item UNIX Betriebssystem,
  \item git,
  \item sbt.
\end{itemize}

\section{Installation}

\begin{enumerate}
  \item Mit git das Repositorium herunterladen:
        \begin{verbatim}
          git clone https://github.com/themerius/ScalTeX-DSL
        \end{verbatim}
  \item In das \verb+ScalTeX-DSL/Scala-NextGen+ Verzeichnis wechseln,
  \item \verb+sbt+ ausführen,
  \item in der sbt-shell den Befehl \verb+run+ eingeben.
  \item (Mit \verb+test+ werden die Testszenarien ausgeführt.)
\end{enumerate}

Nach dem \verb+run+ Befehl werden von sbt automatisch die benötigten
Bibliothektsabhänigkeiten und korrekte Scala Version heruntergeladen.
Danach werden die Quellen kompiliert und ausgeführt---es sollte dann eine
\verb+_output/output.html+ Datei entstehen, mit dem Dokument als HTML-Datei,
welche das kleine hier im Dokument entstandene JavaScript-Framework verwendet.

In \verb+src/main/scala/main.scala+ liegt das
beispielhafte Dokumentenskript.