\chapter{Zusammenfassung}


\section{Problemstellung}

Erstellung qualitativ hochwertiger Dokumente war,
ist und bleibt eine Herausforderung.
Das hier erforschte Textsatzsystem versucht die ubiquitäre und hoch entwickelte
Webtechnologie für die Darstellung klassischer Dokumente bzw. Printdokumente
zu verwenden.
Mit Vorteilen ausgerüstet wie hoher gestalterischer Freiheit,
interaktiven Inhalten und aktuellster Technik ist Webtechnologie
mittlerweile\footnote{
HTML5-Technologien haben eine immense Qualitätssteigerung einher gebracht.}
eine sehr gute Basis zur Darstellung von hochwertigen Dokumenten.


\section{Ziel}

Mithilfe einer DSL soll der Dokumentenersteller eine intuitive Methode
bekommen, um ein Dokument auf Basis von Webtechnologie zu setzen.
Der Klassiker \TeX~kann auch als eine DSL zur Dokumentengenerierung
angesehen werden und dient als Inspiration.

Bevor jedoch das Textsatzsystem implementiert werden kann, muss eine
geeignete DSL-Technologie gefunden werden.
Dazu wurden die zwei DSL-Technologien, Details in Kapitel \ref{sec-vergleich},
verglichen:
Das Xtext-Framework zur Erstellung \emph{externer DSLs} und
die Scala-Programmiersprache zur Erstellung \emph{interner DSLs}.


\section{Vorgehen}

Zum Vergleich der zwei DSL-Technologien sind
in der Vergleichsmatrix in Kapitel \ref{sec-vergleichsmatrix}
15 Fähigkeiten aufgeführt, jede Fähigkeit enhält je eine Beschreibung
wo Xtext mit Scala gegenübergestellt wird.

Die Zielarchitektur ist Webtechnologie, für diese ein JavaScript-Framework
entwickelt wurde, welches eine Abstraktion für Seiten einführt, so dass eine
Webseite Eigenschaften und Aussehen eines Printdokument annehmen kann.

Es konnte gezeigt werden,
dass eine intuitive DSL zur Dokumentengenerierung, inspiriert von \TeX,
auch mit Scala vernünftig umsetzbar ist---ein ausführliches Beispiel
für ein DSL-Skript ist im Anhang \ref{sec-api-resultat} und
nähere Erläuterungen dazu in Kapitel \ref{sec-api-design}.


\section{Ergebnisse}

\paragraph{Vergleich.}
Mit Scala als Gewinner des Vergleichs wird der DSL-Teil des Textsatzsystems als
interne DSL implementiert. Da eine interne Scala-DSL einer Softwarebibliothek
entspricht, sind somit alle Scala-Programmierfähigkeiten im DSL-Skript nutzbar.

Scala hat eine starke Sprachinfrastruktur insbesondere liefert Scala eine
ausgereifte Code-Strukturierung mit.

Scala kann es einem DSL-Benutzer ermöglichen, selbst Dokumententemplates
zu schreiben.

Elegante Lösung des Vorwärtsverweisproblems möglich---siehe Kapitel
\ref{sec-forwardreference}.

Dieser Vergleich kann auch allgemeinen für \emph{externe versus
interne DSLs} gelten, wenn die Xtext- bzw. Scala-spezifischen Besonderheiten
außen vor gelassen werden.
Da das Ergebnis auf den Anwendungsfall „Textsatzsystem“ getrimmt ist muss
ggf. eine Anpassung, insbesondere bei der Bewertung, vorgenommen werden,
falls die Ergebnisse für andere Projekte adaptiert werden sollen.

\paragraph{Zielarchitektur.}
Um eine Webseite wie ein Printdokument aussehen zu lassen wurde ein
JavaScript-Framework erstellt, welches folgende Ziele erreicht:

\erreichtZielarchi

Während der Entwicklung entstanden u.a. diese Dokumententemplates: (1)
der Fraunhofer Bericht und (2) die Patentschrift des europäischen
Patentregisters.

\paragraph{DSL und Generator.} 
Es wurde eine Scala-DSL und ein eingebauter Generator entwickelt,
die folgenden Ziele umsetzt werden:

\erreichtDSL

\emph{Anmerkung:}
Dieses System wird vom Fraunhofer SCAI für die Aufbereitung von
chemischen Patentschriften eingesetzt.


\section{Ausblick}

Dieses Projekt kann in viele Richtungen weiterentwickelt werden werden,
wie in Abschnitt \ref{sec-idee} näher erläutert---hier einige Beispiele:

\begin{itemize}
  \item Die DSL kann erweitert werden,
        um eine Syntax für z.B. Tabellen zu schaffen.
  \item Weitere Logik Klassen die z.B. ermöglichen verschiedene Formate wie
        Markdown zu importieren und intern in Entitäten umzuwandeln.
  \item Kommunikation mit einem Webserver (via JavaScript) für mehr Interaktion,
        z.B. für Benutzerkommentare im Dokument oder Daten die sich automatisch
        aktualisieren.
  \item Neuartige Templates, die z.B. ein Dokument als Mindmap
        darstellen lassen oder aus dem Dokument eine Präsentation erstellen.
  \item Die Templates brauchen noch striktere Paradigmen, falls mehr
        Templates gewünscht werden, so dass sie leicht austauschbar bleiben.
        Beispielsweise jedes Template hat ein vier Spaltenlayout und n definierte
        Zeilen.
\end{itemize}