\chapter{Zusammenfassung}

Im Verlaufe dieses Dokuments konnte gezeigt werden, dass DSLs eine sehr
spannende Technologie sind um das Problemfeld der automatischen
Dokumentengenerierung bzw. Textsatz anzugehen.

Die externe DSL-Technologie, die das Xtext-Framework bietet,
wurde den Fähigkeiten der Scala-Programmiersprache als interne DSL-Technologie gegenübergestellt und verglichen,
mit dem Ergebnis, dass Scala für den in diesem Dokument ausgesuchten
Anwendungsfall, der Erstellung eines Textsatzsystems,
leicht besser geeignet ist.

Die dabei entstandene interne Scala DSL erlaubt es einem Benutzer ohne große Programmierfähigkeiten, ein Dokumentenskript zu verstehen, erweitern
bzw. zu erstellen.

Dadurch dass es als interne Scala DSL gelöst ist, entsteht der Vorteil,
dass die DSL quasi einer ordinären Scala Bibliothek entspricht und
somit innerhalb des Dokumentenskripts, bei Bedarf,
auf die komplette Mächtigkeit von Scala
zurückgegriffen werden kann---also u.a. eine riesige Auswahl an verschiedenen
Softwarebibliotheken und eine erwachsene Scala-Infrastruktur.

Zudem konnte gezeigt werden, dass Webtechnologie durchaus tauglich ist, um
„klassische“ Dokumente darzustellen. Auch wenn dafür etwas
JavaScript-Programmierarbeit von Nöten ist. Der Vorteil den man sich mit
Webtechnologie einkauft, ist u.a. eine sehr hohe Flexibilität bei der
Gestaltung der Layouts und Inhalte, interaktive Elemente sind naturgegeben
und wenn das Web-Template sich an die W3C-Empfehlungen hält, ist es
standardisiert, was eine Langlebigkeit der entstandenen Dokumente garantiert.

\section{Ausblick}

An diesem Projekt kann noch sehr viel Weiterentwickelt werden, in alle
möglichen Richtungen:

\begin{itemize}
  \item Die DSL kann erweitert werden,
        um eine Syntax für z.B. Tabellen zu schaffen.
  \item Weitere Logik Klassen die z.B. ermöglichen verschiedene Formate wie
        Markdown zu importieren und intern in Entitäten umzuwandeln.
  \item Weiterentwicklung der Webtechnologie-Darstellung, insbesondere
        sollte noch ein Algorithmus der Entitäten, wenn nötig,
        weiter aufteilt um so eine möglichst hohe Auslastung der Seiten
        zu ermöglichen.
  \item Kommunikation mit einem Webserver (via JavaScript) für mehr Interaktion,
        z.B. für Benutzerkommentare im Dokument oder Daten die sich automatisch
        aktualisieren.
  \item Neuartige Templates, die z.B. ein Dokument leicht als Mindmap
        darstellen lassen oder aus dem Dokument eine Präsentation erstellen.
  \item Die Templates brauchen noch striktere Paradigmen, falls mehr
        Templates gewünscht werden, so dass sie leicht austauschbar bleiben.
        Beispielsweise jedes Template hat ein vier Spaltenlayout und n definierte
        Zeilen.
\end{itemize}